% Copyright (c) 2012-2018 by the GalSim developers team on GitHub
% https://github.com/GalSim-developers
%
% This file is part of GalSim: The modular galaxy image simulation toolkit.
% https://github.com/GalSim-developers/GalSim
%
% GalSim is free software: redistribution and use in source and binary forms,
% with or without modification, are permitted provided that the following
% conditions are met:
%
% 1. Redistributions of source code must retain the above copyright notice, this
%    list of conditions, and the disclaimer given in the accompanying LICENSE
%    file.
% 2. Redistributions in binary form must reproduce the above copyright notice,
%    this list of conditions, and the disclaimer given in the documentation
%    and/or other materials provided with the distribution.
%

\documentclass[preprint]{aastex}

% packages for figures
\usepackage{graphicx}
% packages for symbols
\usepackage{latexsym,amssymb}
% AMS-LaTeX package for e.g. subequations
\usepackage{amsmath}

%=====================================================================
% FRONT MATTER
%=====================================================================

\slugcomment{Draft \today}

%=====================================================================
% BEGIN DOCUMENT
%=====================================================================

\newcommand{\beq}{\begin{equation}}
\newcommand{\eeq}{\end{equation}}
\newcommand{\shera}{{\sc shera}}
\newcommand{\sbp}{SBProfile}
\newcommand{\mxx}{\ensuremath{M_{xx}}}
\newcommand{\myy}{\ensuremath{M_{yy}}}

\begin{document}

\title{Comparison between \shera\ and \sbp}

\begin{abstract}
This document describes the results of tests of \shera\ and \sbp\ to
ensure that (a) they give consistent results for realistic galaxies
and PSFs and (b) they are applying the shears sufficiently accurately
for the purposes of the GREAT3 challenge.
\end{abstract}

\section{Motivation}

The \shera\ software was designed specifically to simulate 
ground-based data using realistic training data from HST, and was
written in IDL.  Recently \sbp, which was designed as a more
general-purpose image simulator, was extended to include \shera-like
functionality using the SBInterpolatedImage class to represent general, non-parametric galaxy and
PSF images.  We
want to ensure that the two codes give consistent results, and that
the shears are applied to sufficiently high accuracy for use in GalSim
and GREAT3.

A crucial step where differences can be introduced is interpolation.
There are several steps where interpolation is needed: interpolation
between pixels in real space, and in Fourier
space for the shearing.   These two interpolations need not be done in
the same way; SBProfile allows different interpolation kernels for
real- and Fourier-space operations.

\subsection{The approaches}

There are a few differences between the approaches in \shera\ and in
\sbp\ when mapping a COSMOS galaxy to a sheared one at lower
resolution.  

\begin{itemize}
\item The main difference is that \sbp\ supports any ratio of output
  to input pixel scale, which requires it to interpolate the target
  PSF in Fourier space.  In contrast, \shera\ has more restrictive
  rules about the ratio of output to input pixel scale (requiring that
  $5\times$ (output pixel scale)$/$(input pixel scale) be an integer),
  but this requirement allows the code to avoid explicitly
  interpolating the target PSF in real space.  Instead, it is implicitly
  interpolated, via padding in Fourier space, which is equivalent to
  the ideal sinc interpolation in real space.  Clearly, we require the
  more general \sbp\ treatment for GalSim, and it will be interesting
  to compare these different approaches to the 
  interpolation of the PSFs.
\item \sbp\ allows (and requires) a choice of interpolation kernel in real
  space.  All Fourier operations in \sbp\ account for the fact that
  the real-space image will be interpolated and should {\em not} be wrapped.
\item For the interpolation that is used for the Fourier-space
  shearing step, \shera\ uses a specific interpolation kernel, cubic interpolation (a windowed
appoximation to a sinc based on cubic polynomials, using 16 points for
a 2d image). \sbp\ defaults to a quintic polynomial $6\times6$ pixel
interpolant, but allows the choice of other interpolation kernels.
See the document by Bernstein \& Gruen (2012) for detailed discussion
of the impact of interpolation errors and padding on the accuracy of
shearing and resampling via DFTs.
\item \sbp\ will zero-pad
  images to square linear dimensions $2^n$ or $3\times 2^n$ for efficient FFTs.
\shera\ has rather
  more arcane rules that require the galaxy image and the target PSF
  to be related by a factor of output vs. input pixel scale (to avoid
  having to interpolate the target PSF at all).
\item \shera\ suffers from some annoying centroiding issues
  due to the use of IDL convolution routines.
\item \shera\ and \sbp\ have different flux normalization conventions
  due to their different purposes.  In particular, \sbp\ seeks to
  preserve the surface brightness whereas \shera\ asks the user to specify a target
  flux normalization.
\item \shera\ is carrying out a pseudo-deconvolution whereas \sbp\ is
  carrying out a full deconvolution.  This should not matter as long
  as the pseudo-deconvolution properly looks like a deconvolution for
  all $k$ that are not killed off by the target PSF.
\item The final resampling (via interpolation or subsampling) takes
  place in Fourier domain for \sbp, in real space for \shera.
\end{itemize}

\subsection{Issues}

Comparison of results between the two codes is complicated by the
different conventions for centroiding and flux normalization, and by
the different rules for array sizes. For our initial tests, to allow for compatible input
image sizes between the codes, we require output vs. input pixel size
$=2^n$.  We chose $n=3$, so the ground-based telescope for which we
carry out the tests has $0.24$ arcsec pixels.  (Note that we still use
SDSS PSFs when we need real ones.  We are simply lying to the code
about the pixel size for those PSFs, for our own convenience.)  Also,
since output vs. input pixel sizes are an integer, the final
resampling to the target pixel size is simply a resampling process.
We will eventually want to test the effects of resampling via
interpolation, but that test can take place at a later stage.

\section{The PSF test and non-sheared galaxy tests}

We begin with a test for which we know the answer: we choose some
typical COSMOS PSF as our ``galaxy,'' so that when we carry out
PSF-matching, the final ``galaxy'' image should equal the target PSF.
This test does not involve shearing, it only tests PSF-matching for
rather arbitrary images.  We also carry out the PSF-matching for three
sample galaxies, where the true answer is not known.  

For this test, we compare three types of outputs: we compare the final
image on the target pixel scale; and we compare it on the original pixel
scale (after PSF-matching, which includes the target pixel response
function).  Since the ratio of pixel scales is an integer, the
first image comes from strict subsampling of the second image.

As an aside, we also compare a real-space version of the target PSF
interpolated to the original COSMOS pixel scale, as a test of our
interpolators. This test 
suggests that our cubic convolution implementations do essentially the
same thing: the moments of the interpolated PSFs agree at the
$10^{-5}$ level.

In the initial version of the 1st and 2nd tests (final simulated image
on target and initial pixel scales), we found that the first test gave
results that agree down to the $3\times 10^{-4}$ level for the sum of
$\mxx+\myy$, and $3 \times 10^{-6}$ in ellipticity -- but the second
test gave differences of order a few percent, with \sbp\ outputs being
larger and rounder than the \shera\ outputs.  

Gary's theory is that
this relates to the fact that his Fourier-domain treatment includes
operations to represent the interpolation of the
target PSF.  The cubic interpolation effectively broadens the PSF
slightly, so \sbp\ gives broader outputs at the initial pixel scale,
but not at the final pixel scale (where the resampling essentially
means there is no interpolation).  To test the theory that
this difference comes from the \sbp\ inclusion of the cubic
interpolant as part of the definition of the PSF, Gary replaced the
cubic interpolation of the target PSF with an $N=7$ Lanczos filter.
This filter should more closely approximate what \shera\ does, i.e. no
interpolation of the target PSF, but rather padding and embedding it
into some larger array, which is equivalent to sinc interpolation.
When he does this, the $\mxx+\myy$ from \shera\ and \sbp\ at the
initial COSMOS pixel scale agree to a few $\times 10^{-4}$, and the
ellipticities agree to the fifth decimal place.  

Conclusion for GalSim: we need to keep in mind that interpolation of
the target PSF (either implicitly via FTing and padding, or explicitly
in real or Fourier space) can amount to a different definition of the
PSF, and we must decide what definition to use. Ultimately it comes down to assumptions about what
astronomical objects look like between the samples that make up the pixelated images.

\section{Shear}

We have to test the accuracy of applied shears.  There is no serious
reason to expect the methods to disagree with each other (given that
we established that they carry out cubic interpolation nearly
identically), but this test is still important, particularly a test of
absolute rather than relative accuracy.

\subsection{Accuracy with respect to known analytic model}

We have carried out tests using Gaussians, again adopting the initial and final pixel scales of
0.03\arcsec\ and 0.24\arcsec\ so we do not have to worry about differences due to how resampling is
carried out.  Our Gaussian PSFs have FWHM of 3.5 pixels, and the Gaussian galaxy has a FWHM twice
that of the final target PSF.  We considered shears of $(0, 0)$, $(0.02, 0.)$ and $(0., 0.05)$, and
checked our simulated ground-based images (sheared and convolved with target PSF) versus our
expectations for what the moment matrix should look like.

For the unsheared case, \sbp\ gives a final simulated images with moments \mxx\ and \myy\ that agree
with the ideal case at the $-0.06$\% and $-0.03$\% level for Lanczos
5th and 9th order real-space interpolation
kernels, respectively (used when interpolating the target PSF).  The agreement was worse when using
the cubic interpolation kernel, as seen in tests in previous sections.  \shera\ gives moments \mxx\
and \myy\ that agree with the ideal case to better than the $10^{-6}$ level, due to the fact that
the PSF is effectively interpolated with a sinc.

For the sheared case, \sbp\ gives ellipticities that are accurate to 5 parts in $10^4$ for
Lanczos-5, 2 parts in $10^4$ for Lanczos-9.  \shera\ results initially suggested some substantial
error, but it turns out that this came from the relatively small postage stamp sizes used here
(which is different from an actual use case for \shera, since the COSMOS postage stamps that are
provided have large amounts of padding).  When the postage stamp sizes were increased to provide a
level of padding comparable to the real training data, the errors in
the final ellipticity were $\sim2$ parts in $10^4$.

\subsection{Relative accuracy}

Real galaxies, \sbp\ vs. \shera: in progress.  The value in this comparison compared to the previous
is that shearing realistic galaxy profiles might be harder than shearing nice simple Gaussians.

\section{Resampling}

Once we understand the results of the above tests, we may want to do some tests that 
require resampling to non-integer ratio pixel scales, though I think we do have a basic
understanding of what is going on here.

\section{Outstanding issues and conclusions}

As we found in the first test, we need to be sure to understand the impact of
resampling with a given interpolant in Fourier space (which gives
errors periodic in $k$, and real-space ghosts) or real space (which
gives errors periodic in $x$, and Fourier-space ghosts), and the
impact on the effective PSF.  At some level, this will come down to arbitrary decisions about what
the objects we are observing look like between/beyond the resampled pixel grid.  An important point
to keep in mind, for perspective, is that there are systematic limitations in the training data that
may be more important than the low-level differences we see here.  The main (but not only) such
limitations include (1) the fact that we use MultiDrizzle outputs, which include some aliasing (see
work by Fruchter 2011, or Rowe, Hirata, \& Rhodes 2011 on this topic; it arises because the undersampled individual exposures must be
interpolated), and (2) the use of Tiny Tim PSFs, which do not perfectly represent the real ACS
PSFs.  For GREAT3, we are likely to circumvent (2) by carrying out a PCA modeling of real stars in
the ACS images, but we will almost certainly not do anything about (1). 

The critical issue for GalSim is not so much that the galaxies look
exactly like the real galaxies on the sky (as long as they do not look
so different that size- and profile-dependent biases are significantly
affected). Rather, GalSim must draw
sheared galaxies that are very precisely equivalent to sheared
versions of the un-sheared galaxies that it draws.  In this respect
our goal is that the shear inferred from GalSim images be accurate to
well below 1 part in $10^3$.  Since the choice of real-space
interpolant only affects the former, not the latter issue, any
consistent choice should be sufficient.  Tests so far indicate that
\sbp\ and \shera\ agree and are both correct on shear at the required
level, as long as sufficient zero-padding is done before the FFTs.

A point that Gary raised to consider for the future: are our images
giving surface brightness (the assumption of \sbp) or flux per pixel
(the assumption of \shera, which means to match to some known object
flux in ground-based data)?  And when we add magnification to the
pixel, do we want to conserve flux or surface brightness?  (\sbp\ does
the latter.  We should make this choice clear and decide whether it
makes sense for us for GalSim.  I think it does.)

\section{Conclusions and decisions to be made}

The above comparison suggests that provided we choose carefully what interpolants to use, \sbp\
should allow us to simulate sheared, realistic galaxies at the level of precision we need.

There are several places where interpolation must occur, and it's not clear that the same level of
accuracy is needed for each.  In summary, these places are:
\begin{itemize}
\item Interpolation of the target PSF to get a higher resolution version.
\item Shearing.
\item Resampling.
\end{itemize}

For interpolation of the target PSF, we can see the effects in the tests above with Gaussians.  5th
order Lanczos gives moments that are accurate to $5\times 10^{-4}$, or linear sizes that are
accurate to $\sim 2\times 10^{-4}$.  Given the aforementioned uncertainties in the training data
(due to MultiDrizzle, Tiny Tim) I think it does not make sense to optimize for even greater accuracy
than that in the sizes.  It is arguable whether even this is necessary, and we can consider less
accurate interpolations such as quintic for this stage (though we already know that cubic
interpolants give differences of order a few \%, which seems dicey).

For shearing, where we would like our shears to be accurate at the $10^{-4}$ level, it seems that we
might require 9th order Lanczos in \sbp.  There might be a little room to change the order (should be $>5$),
but certainly cubic/quintic interpolation kernels are inadequate.

For resampling, is further testing required?  The conclusions are likely the same as for the
interpolation of the target PSF (rather than the more stringent requirements for shearing).

\end{document}
