%% Generated by Sphinx.
\def\sphinxdocclass{report}
\documentclass[a4paper,10pt,english]{sphinxmanual}
\ifdefined\pdfpxdimen
   \let\sphinxpxdimen\pdfpxdimen\else\newdimen\sphinxpxdimen
\fi \sphinxpxdimen=.75bp\relax

\PassOptionsToPackage{warn}{textcomp}
\usepackage[utf8]{inputenc}
\ifdefined\DeclareUnicodeCharacter
 \ifdefined\DeclareUnicodeCharacterAsOptional
  \DeclareUnicodeCharacter{"00A0}{\nobreakspace}
  \DeclareUnicodeCharacter{"2500}{\sphinxunichar{2500}}
  \DeclareUnicodeCharacter{"2502}{\sphinxunichar{2502}}
  \DeclareUnicodeCharacter{"2514}{\sphinxunichar{2514}}
  \DeclareUnicodeCharacter{"251C}{\sphinxunichar{251C}}
  \DeclareUnicodeCharacter{"2572}{\textbackslash}
 \else
  \DeclareUnicodeCharacter{00A0}{\nobreakspace}
  \DeclareUnicodeCharacter{2500}{\sphinxunichar{2500}}
  \DeclareUnicodeCharacter{2502}{\sphinxunichar{2502}}
  \DeclareUnicodeCharacter{2514}{\sphinxunichar{2514}}
  \DeclareUnicodeCharacter{251C}{\sphinxunichar{251C}}
  \DeclareUnicodeCharacter{2572}{\textbackslash}
 \fi
\fi
\usepackage{cmap}
\usepackage[T1]{fontenc}
\usepackage{amsmath,amssymb,amstext}
\usepackage{babel}
\usepackage{times}
\usepackage[Bjarne]{fncychap}
\usepackage{sphinx}

\usepackage{geometry}

% Include hyperref last.
\usepackage{hyperref}
% Fix anchor placement for figures with captions.
\usepackage{hypcap}% it must be loaded after hyperref.
% Set up styles of URL: it should be placed after hyperref.
\urlstyle{same}

\addto\captionsenglish{\renewcommand{\figurename}{Fig.}}
\addto\captionsenglish{\renewcommand{\tablename}{Table}}
\addto\captionsenglish{\renewcommand{\literalblockname}{Listing}}

\addto\captionsenglish{\renewcommand{\literalblockcontinuedname}{continued from previous page}}
\addto\captionsenglish{\renewcommand{\literalblockcontinuesname}{continues on next page}}

\addto\extrasenglish{\def\pageautorefname{page}}

\setcounter{tocdepth}{1}


        \usepackage{svg}
        \usepackage{charter}
        \usepackage[defaultsans]{lato}
        \usepackage{inconsolata}
    

\title{MR SPRINT Documentation}
\date{Jun 27, 2018}
\release{1.2}
\author{Daniel Cosmo Pizetta, Victor Hugo de Mello Pessoa}
\newcommand{\sphinxlogo}{\sphinxincludegraphics{MRSprintLogo.png}\par}
\renewcommand{\releasename}{Release}
\makeindex

\begin{document}

\maketitle
\sphinxtableofcontents
\phantomsection\label{\detokenize{index::doc}}


\noindent{\hspace*{\fill}\sphinxincludegraphics[scale=1.0]{{MRSprintLogo}.png}\hspace*{\fill}}

\sphinxhref{https://pypi.python.org/pypi/mrsprint}{\sphinxincludegraphics[scale=1.0]{{mrsprint}.png}} \sphinxhref{https://mrsprint.readthedocs.io/en/latest/?badge=latest}{}


\chapter{Welcome!}
\label{\detokenize{readme:welcome}}\label{\detokenize{readme::doc}}
\sphinxstylestrong{Magnetic resonance experiment simulator and visualization tool}

MRSPRINT is a visual magnetic resonance simulator where you can simulate
a magnetic resonance experiment - spectroscopy or imaging. Its main goal is
to be an education tool that assists the student/staff to understand,
interpret and explore magnetic resonance phenomena.

This tool is totally free (see license), but if you are making use of it,
you need to cite us using both citations. This is very important for us.
\begin{itemize}
\item {} 
Article: Coming soon!

\item {} 
Software: Coming soon!

\end{itemize}

If you are using this piece of software to generate images, gif’s, movies,
etc., or using images available on this site, please, also reference us
using those citations.


\section{What can you see?}
\label{\detokenize{readme:what-can-you-see}}\begin{itemize}
\item {} 
Precession: spins precessing in static magnetic field;

\item {} 
Resonance: resonance when a RF pulse is applied;

\item {} 
Contrast: T1, T2, and density of spins;

\item {} 
Field inhomogeneity: isochromates can be shown with their dispersion;

\item {} 
Gradient: magnetic field gradient in action, its intensity and effect over the frequency;

\item {} 
Evolution: over magnetization with intensity, frequency and phase and the pulse sequence;

\item {} 
FID: free induction decay, the signal;

\item {} 
Echo: spin or gradient echo (rephasing/dephasing).

\end{itemize}


\section{Future planned features}
\label{\detokenize{readme:future-planned-features}}\begin{itemize}
\item {} 
K-space visualization;

\item {} 
Multi-nuclei samples/experiments;

\item {} 
T2* as sample parameter to easily setup field inhomogeneity;

\item {} 
Graphical sequence editor;

\item {} 
Chemical interactions;

\item {} 
View on coordinate laboratory system;

\item {} 
Flow (spins not fixed in positions).

\end{itemize}


\section{Documentation}
\label{\detokenize{readme:documentation}}
\sphinxhref{https://mrsprint.readthedocs.io/en/latest/}{Go to documentation on ReadTheDocs!}
It is available on Read The Docs in HTML, EPUB, and PDF.


\section{Download binaries - click-and-run}
\label{\detokenize{readme:download-binaries-click-and-run}}
Binaries are for those do not wish to install any Python things.
We recommend them to the ones without any programming experience.
Download from links below.
\begin{itemize}
\item {} 
Portable Windows Binaries: coming soon!

\item {} 
Portable Linux Binaries: coming soon!

\item {} 
Portable Mac Binaries: coming soon!

\end{itemize}

Sou you can just download, decompress, click-and-run.


\section{Installing from PyPI - stable, end-user}
\label{\detokenize{readme:installing-from-pypi-stable-end-user}}
To install, do

\sphinxcode{\sphinxupquote{\$ pip install mrsprint}}

This will install all necessary dependencies then the code.

To run from terminal

\sphinxcode{\sphinxupquote{\$ mrsprint}}


\section{Dependencies}
\label{\detokenize{readme:dependencies}}\begin{itemize}
\item {} 
NumPy: Numerical mathematical library;

\item {} 
SciPy: Scientific library;

\item {} 
NMRGlue: NMR processing library;

\item {} 
H5Py: Storing and managing data files;

\item {} 
PyQtGraph: Data visualization library;

\item {} 
PyQt/Pyside: Graphical framework.

\end{itemize}

Dependencies are automatically installed when using the method above.


\chapter{Overview}
\label{\detokenize{overview:overview}}\label{\detokenize{overview::doc}}

\section{Main screens}
\label{\detokenize{overview:main-screens}}
Context editors are specialized to edit each step of experiment, see below.

\noindent{\hspace*{\fill}\sphinxincludegraphics[scale=1.0]{{toolbar-context}.png}\hspace*{\fill}}


\subsection{Sample}
\label{\detokenize{overview:sample}}
You can open and/or create samples to run with MR experiment. Each sample
element have three characteristics: t1,t2 and density of spins.

\noindent{\hspace*{\fill}\sphinxincludegraphics[scale=0.7]{{screen-sample}.png}\hspace*{\fill}}


\subsection{System}
\label{\detokenize{overview:system}}
Here you can set static magnetic field and add inhomogeneity to it.

\noindent{\hspace*{\fill}\sphinxincludegraphics[scale=0.7]{{screen-system}.png}\hspace*{\fill}}


\subsection{Sequence}
\label{\detokenize{overview:sequence}}
You can choose a pulse sequence for your experiment, that includes the RF
and gradient pulses. The sequence is programmed in Python at this moment.

\noindent{\hspace*{\fill}\sphinxincludegraphics[scale=0.7]{{screen-sequence}.png}\hspace*{\fill}}


\subsection{Simulator}
\label{\detokenize{overview:simulator}}
At this point you can set simulation mode and other details about
simulation , e.g. time resolution.

\noindent{\hspace*{\fill}\sphinxincludegraphics[scale=0.7]{{screen-simulator}.png}\hspace*{\fill}}


\subsection{Processing}
\label{\detokenize{overview:processing}}
Finally you can process your data to plot the spectrum, imaging or
other features.

\noindent{\hspace*{\fill}\sphinxincludegraphics[scale=0.7]{{screen-processing}.png}\hspace*{\fill}}


\section{2D Editor}
\label{\detokenize{overview:d-editor}}
This editor provides a table that represents the current selected
slice of the 3D view. In this table you can edit the values of each
element property(ies). Colors also help you to pre visualize the intensity
of chosen value.


\section{3D View}
\label{\detokenize{overview:d-view}}
3D view is used to show contexts objects such as sample, magnet field
inhomogeneity, and the evolution of magnetization. You can move the cam
to adjust the perspective.


\section{Data}
\label{\detokenize{overview:data}}
Data can be saved using HDF5 files. For each context we adopted an
extension to distinguish from each other. HDF5 files can be easily
modified with other external tool if necessary.


\chapter{History}
\label{\detokenize{history:history}}\label{\detokenize{history::doc}}

\section{A brief history of this universe}
\label{\detokenize{history:a-brief-history-of-this-universe}}
Being short, when difficulties arises at understanding magnetic
resonance phenomena, we should search tools that could help us.

At that time we have found nice explanations on
\sphinxhref{http://mrsrl.stanford.edu/~brian/bloch/}{Brian site}.

As we are fan of Python, we discovered this implementation from
\sphinxhref{https://github.com/neji49/bloch-simulator-python}{Neji}.

Yet, it was difficult to change parameters and create new graphics.
So, this work starts. From that, many ideas have shining, and, here we are.

The main developer/maintainer is the PhD in computational physics
Daniel C. Pizetta, at University of São Paulo, São Carlos, Brazil.

A lot of the code was developed by the bachelor, in physics and
biomolecular sciences, Victor H.M. Pessoa in his final paper.
He created many new functionalities, implemented
all file interactions and connections between graphical interface and
core code.

From now on, Clara Vidor is continuing the development, and we hope that
will still new features to be released soon.

We also have a head professor PhD Fernando Fernandes Paiva that provides
a technical support in magnetic resonance issues.


\chapter{Reference}
\label{\detokenize{modules:reference}}\label{\detokenize{modules::doc}}

\section{examples}
\label{\detokenize{autodoc/examples/modules:examples}}\label{\detokenize{autodoc/examples/modules::doc}}

\section{mrsprint}
\label{\detokenize{autodoc/mrsprint/modules:mrsprint}}\label{\detokenize{autodoc/mrsprint/modules::doc}}

\subsection{mrsprint package}
\label{\detokenize{autodoc/mrsprint/mrsprint:mrsprint-package}}\label{\detokenize{autodoc/mrsprint/mrsprint::doc}}

\subsubsection{Subpackages}
\label{\detokenize{autodoc/mrsprint/mrsprint:subpackages}}

\paragraph{mrsprint.gui package}
\label{\detokenize{autodoc/mrsprint/mrsprint.gui:mrsprint-gui-package}}\label{\detokenize{autodoc/mrsprint/mrsprint.gui::doc}}

\subparagraph{Submodules}
\label{\detokenize{autodoc/mrsprint/mrsprint.gui:submodules}}

\subparagraph{mrsprint.gui.mrsprint\_rc module}
\label{\detokenize{autodoc/mrsprint/mrsprint.gui:module-mrsprint.gui.mrsprint_rc}}\label{\detokenize{autodoc/mrsprint/mrsprint.gui:mrsprint-gui-mrsprint-rc-module}}\index{mrsprint.gui.mrsprint\_rc (module)}\index{qCleanupResources() (in module mrsprint.gui.mrsprint\_rc)}

\begin{fulllineitems}
\phantomsection\label{\detokenize{autodoc/mrsprint/mrsprint.gui:mrsprint.gui.mrsprint_rc.qCleanupResources}}\pysiglinewithargsret{\sphinxcode{\sphinxupquote{mrsprint.gui.mrsprint\_rc.}}\sphinxbfcode{\sphinxupquote{qCleanupResources}}}{}{}
\end{fulllineitems}

\index{qInitResources() (in module mrsprint.gui.mrsprint\_rc)}

\begin{fulllineitems}
\phantomsection\label{\detokenize{autodoc/mrsprint/mrsprint.gui:mrsprint.gui.mrsprint_rc.qInitResources}}\pysiglinewithargsret{\sphinxcode{\sphinxupquote{mrsprint.gui.mrsprint\_rc.}}\sphinxbfcode{\sphinxupquote{qInitResources}}}{}{}
\end{fulllineitems}



\subparagraph{mrsprint.gui.mw\_gradient module}
\label{\detokenize{autodoc/mrsprint/mrsprint.gui:module-mrsprint.gui.mw_gradient}}\label{\detokenize{autodoc/mrsprint/mrsprint.gui:mrsprint-gui-mw-gradient-module}}\index{mrsprint.gui.mw\_gradient (module)}\index{Ui\_Gradient (class in mrsprint.gui.mw\_gradient)}

\begin{fulllineitems}
\phantomsection\label{\detokenize{autodoc/mrsprint/mrsprint.gui:mrsprint.gui.mw_gradient.Ui_Gradient}}\pysigline{\sphinxbfcode{\sphinxupquote{class }}\sphinxcode{\sphinxupquote{mrsprint.gui.mw\_gradient.}}\sphinxbfcode{\sphinxupquote{Ui\_Gradient}}}
Bases: \sphinxhref{https://docs.python.org/3/library/functions.html\#object}{\sphinxcode{\sphinxupquote{object}}}
\index{retranslateUi() (mrsprint.gui.mw\_gradient.Ui\_Gradient method)}

\begin{fulllineitems}
\phantomsection\label{\detokenize{autodoc/mrsprint/mrsprint.gui:mrsprint.gui.mw_gradient.Ui_Gradient.retranslateUi}}\pysiglinewithargsret{\sphinxbfcode{\sphinxupquote{retranslateUi}}}{\emph{Gradient}}{}
\end{fulllineitems}

\index{setupUi() (mrsprint.gui.mw\_gradient.Ui\_Gradient method)}

\begin{fulllineitems}
\phantomsection\label{\detokenize{autodoc/mrsprint/mrsprint.gui:mrsprint.gui.mw_gradient.Ui_Gradient.setupUi}}\pysiglinewithargsret{\sphinxbfcode{\sphinxupquote{setupUi}}}{\emph{Gradient}}{}
\end{fulllineitems}


\end{fulllineitems}



\subparagraph{mrsprint.gui.mw\_mrsprint module}
\label{\detokenize{autodoc/mrsprint/mrsprint.gui:module-mrsprint.gui.mw_mrsprint}}\label{\detokenize{autodoc/mrsprint/mrsprint.gui:mrsprint-gui-mw-mrsprint-module}}\index{mrsprint.gui.mw\_mrsprint (module)}\index{Ui\_MainWindow (class in mrsprint.gui.mw\_mrsprint)}

\begin{fulllineitems}
\phantomsection\label{\detokenize{autodoc/mrsprint/mrsprint.gui:mrsprint.gui.mw_mrsprint.Ui_MainWindow}}\pysigline{\sphinxbfcode{\sphinxupquote{class }}\sphinxcode{\sphinxupquote{mrsprint.gui.mw\_mrsprint.}}\sphinxbfcode{\sphinxupquote{Ui\_MainWindow}}}
Bases: \sphinxhref{https://docs.python.org/3/library/functions.html\#object}{\sphinxcode{\sphinxupquote{object}}}
\index{retranslateUi() (mrsprint.gui.mw\_mrsprint.Ui\_MainWindow method)}

\begin{fulllineitems}
\phantomsection\label{\detokenize{autodoc/mrsprint/mrsprint.gui:mrsprint.gui.mw_mrsprint.Ui_MainWindow.retranslateUi}}\pysiglinewithargsret{\sphinxbfcode{\sphinxupquote{retranslateUi}}}{\emph{MainWindow}}{}
\end{fulllineitems}

\index{setupUi() (mrsprint.gui.mw\_mrsprint.Ui\_MainWindow method)}

\begin{fulllineitems}
\phantomsection\label{\detokenize{autodoc/mrsprint/mrsprint.gui:mrsprint.gui.mw_mrsprint.Ui_MainWindow.setupUi}}\pysiglinewithargsret{\sphinxbfcode{\sphinxupquote{setupUi}}}{\emph{MainWindow}}{}
\end{fulllineitems}


\end{fulllineitems}



\subparagraph{mrsprint.gui.mw\_settings module}
\label{\detokenize{autodoc/mrsprint/mrsprint.gui:module-mrsprint.gui.mw_settings}}\label{\detokenize{autodoc/mrsprint/mrsprint.gui:mrsprint-gui-mw-settings-module}}\index{mrsprint.gui.mw\_settings (module)}\index{Ui\_Settings (class in mrsprint.gui.mw\_settings)}

\begin{fulllineitems}
\phantomsection\label{\detokenize{autodoc/mrsprint/mrsprint.gui:mrsprint.gui.mw_settings.Ui_Settings}}\pysigline{\sphinxbfcode{\sphinxupquote{class }}\sphinxcode{\sphinxupquote{mrsprint.gui.mw\_settings.}}\sphinxbfcode{\sphinxupquote{Ui\_Settings}}}
Bases: \sphinxhref{https://docs.python.org/3/library/functions.html\#object}{\sphinxcode{\sphinxupquote{object}}}
\index{retranslateUi() (mrsprint.gui.mw\_settings.Ui\_Settings method)}

\begin{fulllineitems}
\phantomsection\label{\detokenize{autodoc/mrsprint/mrsprint.gui:mrsprint.gui.mw_settings.Ui_Settings.retranslateUi}}\pysiglinewithargsret{\sphinxbfcode{\sphinxupquote{retranslateUi}}}{\emph{Settings}}{}
\end{fulllineitems}

\index{setupUi() (mrsprint.gui.mw\_settings.Ui\_Settings method)}

\begin{fulllineitems}
\phantomsection\label{\detokenize{autodoc/mrsprint/mrsprint.gui:mrsprint.gui.mw_settings.Ui_Settings.setupUi}}\pysiglinewithargsret{\sphinxbfcode{\sphinxupquote{setupUi}}}{\emph{Settings}}{}
\end{fulllineitems}


\end{fulllineitems}



\subparagraph{Module contents}
\label{\detokenize{autodoc/mrsprint/mrsprint.gui:module-mrsprint.gui}}\label{\detokenize{autodoc/mrsprint/mrsprint.gui:module-contents}}\index{mrsprint.gui (module)}
Package for GUI related classes and objects.
\begin{description}
\item[{Authors:}] \leavevmode\begin{itemize}
\item {} 
Victor Hugo de Mello Pessoa \textless{}\sphinxhref{mailto:victor.pessoa@usp.br}{victor.pessoa@usp.br}\textgreater{}

\item {} 
Daniel Cosmo Pizetta \textless{}\sphinxhref{mailto:daniel.pizetta@usp.br}{daniel.pizetta@usp.br}\textgreater{}

\end{itemize}

\item[{Since:}] \leavevmode
2017/01/09

\end{description}


\paragraph{mrsprint.sequence package}
\label{\detokenize{autodoc/mrsprint/mrsprint.sequence:mrsprint-sequence-package}}\label{\detokenize{autodoc/mrsprint/mrsprint.sequence::doc}}

\subparagraph{Submodules}
\label{\detokenize{autodoc/mrsprint/mrsprint.sequence:submodules}}

\subparagraph{mrsprint.sequence.protocol module}
\label{\detokenize{autodoc/mrsprint/mrsprint.sequence:mrsprint-sequence-protocol-module}}

\subparagraph{mrsprint.sequence.sequence module}
\label{\detokenize{autodoc/mrsprint/mrsprint.sequence:module-mrsprint.sequence.sequence}}\label{\detokenize{autodoc/mrsprint/mrsprint.sequence:mrsprint-sequence-sequence-module}}\index{mrsprint.sequence.sequence (module)}
Module for pulse sequence related classes and functions.
\begin{description}
\item[{Authors:}] \leavevmode\begin{itemize}
\item {} 
Victor Hugo de Mello Pessoa \textless{}\sphinxhref{mailto:victor.pessoa@usp.br}{victor.pessoa@usp.br}\textgreater{}

\item {} 
Daniel Cosmo Pizetta \textless{}\sphinxhref{mailto:daniel.pizetta@usp.br}{daniel.pizetta@usp.br}\textgreater{}

\end{itemize}

\item[{Since:}] \leavevmode
2017/07/01

\end{description}
\index{CPMGSequence (class in mrsprint.sequence.sequence)}

\begin{fulllineitems}
\phantomsection\label{\detokenize{autodoc/mrsprint/mrsprint.sequence:mrsprint.sequence.sequence.CPMGSequence}}\pysigline{\sphinxbfcode{\sphinxupquote{class }}\sphinxcode{\sphinxupquote{mrsprint.sequence.sequence.}}\sphinxbfcode{\sphinxupquote{CPMGSequence}}}
Bases: {\hyperref[\detokenize{autodoc/mrsprint/mrsprint.sequence:mrsprint.sequence.sequence.Sequence}]{\sphinxcrossref{\sphinxcode{\sphinxupquote{mrsprint.sequence.sequence.Sequence}}}}}

Generates a CPMG sequence.

\end{fulllineitems}

\index{GradientEchoSequence (class in mrsprint.sequence.sequence)}

\begin{fulllineitems}
\phantomsection\label{\detokenize{autodoc/mrsprint/mrsprint.sequence:mrsprint.sequence.sequence.GradientEchoSequence}}\pysigline{\sphinxbfcode{\sphinxupquote{class }}\sphinxcode{\sphinxupquote{mrsprint.sequence.sequence.}}\sphinxbfcode{\sphinxupquote{GradientEchoSequence}}}
Bases: {\hyperref[\detokenize{autodoc/mrsprint/mrsprint.sequence:mrsprint.sequence.sequence.Sequence}]{\sphinxcrossref{\sphinxcode{\sphinxupquote{mrsprint.sequence.sequence.Sequence}}}}}

Generate a Gradient Echo sequence.

\end{fulllineitems}

\index{Sequence (class in mrsprint.sequence.sequence)}

\begin{fulllineitems}
\phantomsection\label{\detokenize{autodoc/mrsprint/mrsprint.sequence:mrsprint.sequence.sequence.Sequence}}\pysiglinewithargsret{\sphinxbfcode{\sphinxupquote{class }}\sphinxcode{\sphinxupquote{mrsprint.sequence.sequence.}}\sphinxbfcode{\sphinxupquote{Sequence}}}{\emph{settings}, \emph{nucleus}, \emph{**opts}}{}
Bases: \sphinxcode{\sphinxupquote{pyqtgraph.parametertree.parameterTypes.GroupParameter}}

Class that represents a sequence for magnetic resonance systems.
\index{getGradient() (mrsprint.sequence.sequence.Sequence method)}

\begin{fulllineitems}
\phantomsection\label{\detokenize{autodoc/mrsprint/mrsprint.sequence:mrsprint.sequence.sequence.Sequence.getGradient}}\pysiglinewithargsret{\sphinxbfcode{\sphinxupquote{getGradient}}}{}{}
Return the entire gradient.

\end{fulllineitems}

\index{getRF() (mrsprint.sequence.sequence.Sequence method)}

\begin{fulllineitems}
\phantomsection\label{\detokenize{autodoc/mrsprint/mrsprint.sequence:mrsprint.sequence.sequence.Sequence.getRF}}\pysiglinewithargsret{\sphinxbfcode{\sphinxupquote{getRF}}}{}{}
Return the entire rf.

\end{fulllineitems}

\index{setGradient() (mrsprint.sequence.sequence.Sequence method)}

\begin{fulllineitems}
\phantomsection\label{\detokenize{autodoc/mrsprint/mrsprint.sequence:mrsprint.sequence.sequence.Sequence.setGradient}}\pysiglinewithargsret{\sphinxbfcode{\sphinxupquote{setGradient}}}{\emph{grad\_value}}{}
Properly sets the gradient.

\end{fulllineitems}

\index{setRF() (mrsprint.sequence.sequence.Sequence method)}

\begin{fulllineitems}
\phantomsection\label{\detokenize{autodoc/mrsprint/mrsprint.sequence:mrsprint.sequence.sequence.Sequence.setRF}}\pysiglinewithargsret{\sphinxbfcode{\sphinxupquote{setRF}}}{\emph{rf\_value}}{}
Properly sets the rf stream signal.

\end{fulllineitems}


\end{fulllineitems}



\subparagraph{Module contents}
\label{\detokenize{autodoc/mrsprint/mrsprint.sequence:module-mrsprint.sequence}}\label{\detokenize{autodoc/mrsprint/mrsprint.sequence:module-contents}}\index{mrsprint.sequence (module)}
Package for sequence related classes and objects.
\begin{description}
\item[{Authors:}] \leavevmode\begin{itemize}
\item {} 
Daniel Cosmo Pizetta \textless{}\sphinxhref{mailto:daniel.pizetta@usp.br}{daniel.pizetta@usp.br}\textgreater{}

\end{itemize}

\item[{Authors:}] \leavevmode
2015/11/01

\end{description}

\begin{sphinxadmonition}{note}{\label{autodoc/mrsprint/mrsprint.sequence:index-0}Todo:}
Fix these modules.
\end{sphinxadmonition}


\paragraph{mrsprint.simulator package}
\label{\detokenize{autodoc/mrsprint/mrsprint.simulator:mrsprint-simulator-package}}\label{\detokenize{autodoc/mrsprint/mrsprint.simulator::doc}}

\subparagraph{Submodules}
\label{\detokenize{autodoc/mrsprint/mrsprint.simulator:submodules}}

\subparagraph{mrsprint.simulator.old\_plot module}
\label{\detokenize{autodoc/mrsprint/mrsprint.simulator:mrsprint-simulator-old-plot-module}}

\subparagraph{mrsprint.simulator.plot module}
\label{\detokenize{autodoc/mrsprint/mrsprint.simulator:module-mrsprint.simulator.plot}}\label{\detokenize{autodoc/mrsprint/mrsprint.simulator:mrsprint-simulator-plot-module}}\index{mrsprint.simulator.plot (module)}
Module for plotting related classes and functions.
\begin{description}
\item[{Authors:}] \leavevmode\begin{itemize}
\item {} 
Victor Hugo de Mello Pessoa \textless{}\sphinxhref{mailto:victor.pessoa@usp.br}{victor.pessoa@usp.br}\textgreater{}

\item {} 
Daniel Cosmo Pizetta \textless{}\sphinxhref{mailto:daniel.pizetta@usp.br}{daniel.pizetta@usp.br}\textgreater{}

\end{itemize}

\item[{Since:}] \leavevmode
2017/07/01

\end{description}
\index{Plot (class in mrsprint.simulator.plot)}

\begin{fulllineitems}
\phantomsection\label{\detokenize{autodoc/mrsprint/mrsprint.simulator:mrsprint.simulator.plot.Plot}}\pysiglinewithargsret{\sphinxbfcode{\sphinxupquote{class }}\sphinxcode{\sphinxupquote{mrsprint.simulator.plot.}}\sphinxbfcode{\sphinxupquote{Plot}}}{\emph{settings}, \emph{mx}, \emph{my}, \emph{mz}, \emph{gr}, \emph{tp}, \emph{freq\_shift}, \emph{position}, \emph{max\_magnetization}}{}
Bases: \sphinxhref{https://docs.python.org/3/library/functions.html\#object}{\sphinxcode{\sphinxupquote{object}}}

Class that contains the plots of simulation.
\begin{quote}\begin{description}
\item[{Parameters}] \leavevmode\begin{itemize}
\item {} 
\sphinxstyleliteralstrong{\sphinxupquote{settings}} ({\hyperref[\detokenize{autodoc/mrsprint/mrsprint:mrsprint.settings.Settings}]{\sphinxcrossref{\sphinxstyleliteralemphasis{\sphinxupquote{Settings}}}}}) \textendash{} Represents the program settings.

\item {} 
\sphinxstyleliteralstrong{\sphinxupquote{mx}} (\sphinxstyleliteralemphasis{\sphinxupquote{np.ndarray}}) \textendash{} Magnetization in x.

\item {} 
\sphinxstyleliteralstrong{\sphinxupquote{my}} (\sphinxstyleliteralemphasis{\sphinxupquote{np.ndarray}}) \textendash{} Magnetization in y.

\item {} 
\sphinxstyleliteralstrong{\sphinxupquote{mz}} (\sphinxstyleliteralemphasis{\sphinxupquote{np.ndarray}}) \textendash{} Magnetization in z.

\item {} 
\sphinxstyleliteralstrong{\sphinxupquote{gr}} (\sphinxstyleliteralemphasis{\sphinxupquote{np.ndarray}}) \textendash{} Gradients of magnetic field.

\item {} 
\sphinxstyleliteralstrong{\sphinxupquote{tp}} (\sphinxstyleliteralemphasis{\sphinxupquote{np.ndarray}}) \textendash{} Time.

\item {} 
\sphinxstyleliteralstrong{\sphinxupquote{freq\_shift}} (\sphinxstyleliteralemphasis{\sphinxupquote{np.ndarray}}) \textendash{} Frequency shift due to field inhomogeneity.

\item {} 
\sphinxstyleliteralstrong{\sphinxupquote{position}} (\sphinxstyleliteralemphasis{\sphinxupquote{np.ndarray}}) \textendash{} Position of spins.

\item {} 
\sphinxstyleliteralstrong{\sphinxupquote{max\_magnetization}} (\sphinxhref{https://docs.python.org/3/library/functions.html\#float}{\sphinxstyleliteralemphasis{\sphinxupquote{float}}}) \textendash{} Maximum value of magnetization.

\end{itemize}

\end{description}\end{quote}
\index{pause() (mrsprint.simulator.plot.Plot method)}

\begin{fulllineitems}
\phantomsection\label{\detokenize{autodoc/mrsprint/mrsprint.simulator:mrsprint.simulator.plot.Plot.pause}}\pysiglinewithargsret{\sphinxbfcode{\sphinxupquote{pause}}}{}{}
Stop the simulation on plot.

\end{fulllineitems}

\index{play() (mrsprint.simulator.plot.Plot method)}

\begin{fulllineitems}
\phantomsection\label{\detokenize{autodoc/mrsprint/mrsprint.simulator:mrsprint.simulator.plot.Plot.play}}\pysiglinewithargsret{\sphinxbfcode{\sphinxupquote{play}}}{}{}
Start the simulation on plot.

\end{fulllineitems}

\index{plotMagnetization() (mrsprint.simulator.plot.Plot method)}

\begin{fulllineitems}
\phantomsection\label{\detokenize{autodoc/mrsprint/mrsprint.simulator:mrsprint.simulator.plot.Plot.plotMagnetization}}\pysiglinewithargsret{\sphinxbfcode{\sphinxupquote{plotMagnetization}}}{\emph{mag\_win}}{}
Create the graphics of magnetization.
\begin{quote}\begin{description}
\item[{Parameters}] \leavevmode
\sphinxstyleliteralstrong{\sphinxupquote{mag\_win}} (\sphinxstyleliteralemphasis{\sphinxupquote{pg.graphicsWindows.GraphicsWindow}}) \textendash{} Window where plot will be made.

\end{description}\end{quote}

\end{fulllineitems}

\index{plotSpin() (mrsprint.simulator.plot.Plot method)}

\begin{fulllineitems}
\phantomsection\label{\detokenize{autodoc/mrsprint/mrsprint.simulator:mrsprint.simulator.plot.Plot.plotSpin}}\pysiglinewithargsret{\sphinxbfcode{\sphinxupquote{plotSpin}}}{\emph{view}}{}
Create the 3D exhibition of magnetization over the experiment.
\begin{quote}\begin{description}
\item[{Parameters}] \leavevmode
\sphinxstyleliteralstrong{\sphinxupquote{view}} (\sphinxstyleliteralemphasis{\sphinxupquote{pg.graphicsWindows.GraphicsWindow}}) \textendash{} Window where the plot will be made.

\end{description}\end{quote}

\end{fulllineitems}

\index{run() (mrsprint.simulator.plot.Plot method)}

\begin{fulllineitems}
\phantomsection\label{\detokenize{autodoc/mrsprint/mrsprint.simulator:mrsprint.simulator.plot.Plot.run}}\pysiglinewithargsret{\sphinxbfcode{\sphinxupquote{run}}}{}{}
Run the simulation on plot, from zero.

\end{fulllineitems}

\index{update() (mrsprint.simulator.plot.Plot method)}

\begin{fulllineitems}
\phantomsection\label{\detokenize{autodoc/mrsprint/mrsprint.simulator:mrsprint.simulator.plot.Plot.update}}\pysiglinewithargsret{\sphinxbfcode{\sphinxupquote{update}}}{}{}
Update magnetization position vectors and timelines on graphics dynamically.

\end{fulllineitems}


\end{fulllineitems}

\index{plot\_item() (in module mrsprint.simulator.plot)}

\begin{fulllineitems}
\phantomsection\label{\detokenize{autodoc/mrsprint/mrsprint.simulator:mrsprint.simulator.plot.plot_item}}\pysiglinewithargsret{\sphinxcode{\sphinxupquote{mrsprint.simulator.plot.}}\sphinxbfcode{\sphinxupquote{plot\_item}}}{\emph{graphics}, \emph{x}, \emph{y}, \emph{line\_color}, \emph{x\_axis=True}, \emph{x\_label=''}, \emph{x\_unit=''}, \emph{y\_axis=True}, \emph{y\_label=''}, \emph{y\_unit=''}}{}
Create a new plot item in a graph.
\begin{quote}\begin{description}
\item[{Parameters}] \leavevmode\begin{itemize}
\item {} 
\sphinxstyleliteralstrong{\sphinxupquote{graphics}} (\sphinxstyleliteralemphasis{\sphinxupquote{pl.graphicsWindow}}) \textendash{} The GraphicsWindow where the plot will be shown.

\item {} 
\sphinxstyleliteralstrong{\sphinxupquote{x}} (\sphinxstyleliteralemphasis{\sphinxupquote{np.array}}) \textendash{} List of x coordinates.

\item {} 
\sphinxstyleliteralstrong{\sphinxupquote{y}} (\sphinxstyleliteralemphasis{\sphinxupquote{np.array}}) \textendash{} List of y coordinates.

\item {} 
\sphinxstyleliteralstrong{\sphinxupquote{line\_color}} (\sphinxhref{https://docs.python.org/3/library/stdtypes.html\#str}{\sphinxstyleliteralemphasis{\sphinxupquote{str}}}) \textendash{} String containing the color (“r”, “g”, “b”, “w”, for example).

\item {} 
\sphinxstyleliteralstrong{\sphinxupquote{x\_axis}} (\sphinxhref{https://docs.python.org/3/library/functions.html\#bool}{\sphinxstyleliteralemphasis{\sphinxupquote{bool}}}) \textendash{} Visibility of x axis.

\item {} 
\sphinxstyleliteralstrong{\sphinxupquote{x\_label}} (\sphinxhref{https://docs.python.org/3/library/stdtypes.html\#str}{\sphinxstyleliteralemphasis{\sphinxupquote{str}}}) \textendash{} Label of x axis.

\item {} 
\sphinxstyleliteralstrong{\sphinxupquote{x\_unit}} (\sphinxhref{https://docs.python.org/3/library/stdtypes.html\#str}{\sphinxstyleliteralemphasis{\sphinxupquote{str}}}) \textendash{} Unit of x axis.

\item {} 
\sphinxstyleliteralstrong{\sphinxupquote{y\_axis}} (\sphinxhref{https://docs.python.org/3/library/functions.html\#bool}{\sphinxstyleliteralemphasis{\sphinxupquote{bool}}}) \textendash{} Visibility of y axis.

\item {} 
\sphinxstyleliteralstrong{\sphinxupquote{y\_label}} (\sphinxhref{https://docs.python.org/3/library/stdtypes.html\#str}{\sphinxstyleliteralemphasis{\sphinxupquote{str}}}) \textendash{} Label of y axis.

\item {} 
\sphinxstyleliteralstrong{\sphinxupquote{y\_unit}} (\sphinxhref{https://docs.python.org/3/library/stdtypes.html\#str}{\sphinxstyleliteralemphasis{\sphinxupquote{str}}}) \textendash{} Unit of y axis.

\end{itemize}

\item[{Returns}] \leavevmode
Plot.

\item[{Return type}] \leavevmode
pg.graphicsItems.PlotItem

\end{description}\end{quote}

\end{fulllineitems}



\subparagraph{mrsprint.simulator.simulator module}
\label{\detokenize{autodoc/mrsprint/mrsprint.simulator:module-mrsprint.simulator.simulator}}\label{\detokenize{autodoc/mrsprint/mrsprint.simulator:mrsprint-simulator-simulator-module}}\index{mrsprint.simulator.simulator (module)}
Module for simulate the response of the sample to the magnetic signal related classes and functions.
\begin{description}
\item[{Authors:}] \leavevmode\begin{itemize}
\item {} 
Victor Hugo de Mello Pessoa \textless{}\sphinxhref{mailto:victor.pessoa@usp.br}{victor.pessoa@usp.br}\textgreater{}

\item {} 
Daniel Cosmo Pizetta \textless{}\sphinxhref{mailto:daniel.pizetta@usp.br}{daniel.pizetta@usp.br}\textgreater{}

\end{itemize}

\item[{Since:}] \leavevmode
2017/07/01

\end{description}
\index{Simulator (class in mrsprint.simulator.simulator)}

\begin{fulllineitems}
\phantomsection\label{\detokenize{autodoc/mrsprint/mrsprint.simulator:mrsprint.simulator.simulator.Simulator}}\pysiglinewithargsret{\sphinxbfcode{\sphinxupquote{class }}\sphinxcode{\sphinxupquote{mrsprint.simulator.simulator.}}\sphinxbfcode{\sphinxupquote{Simulator}}}{\emph{**opts}}{}
Bases: \sphinxcode{\sphinxupquote{pyqtgraph.parametertree.parameterTypes.GroupParameter}}

Class that represents the simulator.

\end{fulllineitems}



\subparagraph{Module contents}
\label{\detokenize{autodoc/mrsprint/mrsprint.simulator:module-mrsprint.simulator}}\label{\detokenize{autodoc/mrsprint/mrsprint.simulator:module-contents}}\index{mrsprint.simulator (module)}
Package for simulator related classes and objects.
\begin{description}
\item[{Authors:}] \leavevmode\begin{itemize}
\item {} 
Daniel Cosmo Pizetta \textless{}\sphinxhref{mailto:daniel.pizetta@usp.br}{daniel.pizetta@usp.br}\textgreater{}

\end{itemize}

\item[{Since:}] \leavevmode
2018/06/07

\end{description}

\begin{sphinxadmonition}{note}{\label{autodoc/mrsprint/mrsprint.simulator:index-0}Todo:}
Maybe remove old\_plot module.
\end{sphinxadmonition}
\index{calculate\_t2\_star() (in module mrsprint.simulator)}

\begin{fulllineitems}
\phantomsection\label{\detokenize{autodoc/mrsprint/mrsprint.simulator:mrsprint.simulator.calculate_t2_star}}\pysiglinewithargsret{\sphinxcode{\sphinxupquote{mrsprint.simulator.}}\sphinxbfcode{\sphinxupquote{calculate\_t2\_star}}}{\emph{t2}, \emph{freq\_shift}}{}
Returns the value for t2*, considering the range of frequencies.
\begin{quote}\begin{description}
\item[{Parameters}] \leavevmode\begin{itemize}
\item {} 
\sphinxstyleliteralstrong{\sphinxupquote{t2}} (\sphinxhref{https://docs.python.org/3/library/functions.html\#float}{\sphinxstyleliteralemphasis{\sphinxupquote{float}}}\sphinxstyleliteralemphasis{\sphinxupquote{ {[}}}\sphinxstyleliteralemphasis{\sphinxupquote{s}}\sphinxstyleliteralemphasis{\sphinxupquote{{]}}}) \textendash{} T2 value in seconds.

\item {} 
\sphinxstyleliteralstrong{\sphinxupquote{freq\_shift}} (\sphinxhref{https://docs.python.org/3/library/functions.html\#float}{\sphinxstyleliteralemphasis{\sphinxupquote{float}}}\sphinxstyleliteralemphasis{\sphinxupquote{ {[}}}\sphinxstyleliteralemphasis{\sphinxupquote{Hz}}\sphinxstyleliteralemphasis{\sphinxupquote{{]}}}) \textendash{} Frequency shift from resonance - symetric
between zero (in resonance).

\end{itemize}

\item[{Returns}] \leavevmode
T2 star value.

\item[{Return type}] \leavevmode
\sphinxhref{https://docs.python.org/3/library/functions.html\#float}{float} {[}s{]}

\end{description}\end{quote}

\begin{sphinxadmonition}{note}{\label{autodoc/mrsprint/mrsprint.simulator:index-1}Todo:}
Confirm if it is the right way to calculate or if it has a weight function for t2star
Pass freq\_shift as a number?
\end{sphinxadmonition}

\end{fulllineitems}

\index{create\_positions() (in module mrsprint.simulator)}

\begin{fulllineitems}
\phantomsection\label{\detokenize{autodoc/mrsprint/mrsprint.simulator:mrsprint.simulator.create_positions}}\pysiglinewithargsret{\sphinxcode{\sphinxupquote{mrsprint.simulator.}}\sphinxbfcode{\sphinxupquote{create\_positions}}}{\emph{size=(1}, \emph{1}, \emph{1)}, \emph{step=(1.0}, \emph{1.0}, \emph{1.0)}, \emph{offset=(0.0}, \emph{0.0}, \emph{0.0)}, \emph{dtype=\textless{}class 'numpy.float32'\textgreater{}}}{}
Creates array of positions.
\begin{quote}\begin{description}
\item[{Parameters}] \leavevmode\begin{itemize}
\item {} 
\sphinxstyleliteralstrong{\sphinxupquote{size}} (\sphinxhref{https://docs.python.org/3/library/stdtypes.html\#tuple}{\sphinxstyleliteralemphasis{\sphinxupquote{tuple}}}\sphinxstyleliteralemphasis{\sphinxupquote{(}}\sphinxhref{https://docs.python.org/3/library/functions.html\#int}{\sphinxstyleliteralemphasis{\sphinxupquote{int}}}\sphinxstyleliteralemphasis{\sphinxupquote{)}}) \textendash{} Size in x, y and z. The minimum number is one for each axis.

\item {} 
\sphinxstyleliteralstrong{\sphinxupquote{step}} (\sphinxhref{https://docs.python.org/3/library/stdtypes.html\#tuple}{\sphinxstyleliteralemphasis{\sphinxupquote{tuple}}}\sphinxstyleliteralemphasis{\sphinxupquote{(}}\sphinxhref{https://docs.python.org/3/library/functions.html\#float}{\sphinxstyleliteralemphasis{\sphinxupquote{float}}}\sphinxstyleliteralemphasis{\sphinxupquote{)}}) \textendash{} Step for each axis.

\item {} 
\sphinxstyleliteralstrong{\sphinxupquote{offset}} (\sphinxhref{https://docs.python.org/3/library/stdtypes.html\#tuple}{\sphinxstyleliteralemphasis{\sphinxupquote{tuple}}}\sphinxstyleliteralemphasis{\sphinxupquote{(}}\sphinxhref{https://docs.python.org/3/library/functions.html\#float}{\sphinxstyleliteralemphasis{\sphinxupquote{float}}}\sphinxstyleliteralemphasis{\sphinxupquote{)}}) \textendash{} Offset for each axis. If zero, the final vector begins in
zero and ends in size plus offset.

\end{itemize}

\item[{Returns}] \leavevmode
\begin{description}
\item[{Position array in this format {[}{[}gr\_x\_plotx gr\_y\_plotx gr\_z\_plotx …{]}}] \leavevmode
{[}gr\_x\_ploty gr\_y\_ploty gr\_z\_ploty …{]}{[}gr\_x\_plotz gr\_y\_plotz gr\_z\_plotz …{]}{]},
where p is the number of position (from offset to size plus offset).

\end{description}


\item[{Return type}] \leavevmode
np.array

\end{description}\end{quote}

\begin{sphinxadmonition}{note}{\label{autodoc/mrsprint/mrsprint.simulator:index-2}Todo:}
Set physical unit for each args.
\end{sphinxadmonition}

\end{fulllineitems}

\index{frequency\_shift() (in module mrsprint.simulator)}

\begin{fulllineitems}
\phantomsection\label{\detokenize{autodoc/mrsprint/mrsprint.simulator:mrsprint.simulator.frequency_shift}}\pysiglinewithargsret{\sphinxcode{\sphinxupquote{mrsprint.simulator.}}\sphinxbfcode{\sphinxupquote{frequency\_shift}}}{\emph{freq\_shift}, \emph{freq\_step=1.0}, \emph{offset=0.0}, \emph{symetric=True}, \emph{dtype=\textless{}class 'numpy.float32'\textgreater{}}}{}
Generates an array of frequency shift, from -frequency\_shift to +frequency\_shift, between offset.
\begin{quote}\begin{description}
\item[{Parameters}] \leavevmode\begin{itemize}
\item {} 
\sphinxstyleliteralstrong{\sphinxupquote{freq\_shift}} (\sphinxhref{https://docs.python.org/3/library/functions.html\#float}{\sphinxstyleliteralemphasis{\sphinxupquote{float}}}\sphinxstyleliteralemphasis{\sphinxupquote{ {[}}}\sphinxstyleliteralemphasis{\sphinxupquote{Hz}}\sphinxstyleliteralemphasis{\sphinxupquote{{]}}}) \textendash{} Maximum frequency to shift.

\item {} 
\sphinxstyleliteralstrong{\sphinxupquote{freq\_step}} (\sphinxhref{https://docs.python.org/3/library/functions.html\#float}{\sphinxstyleliteralemphasis{\sphinxupquote{float}}}\sphinxstyleliteralemphasis{\sphinxupquote{ {[}}}\sphinxstyleliteralemphasis{\sphinxupquote{Hz}}\sphinxstyleliteralemphasis{\sphinxupquote{{]}}}) \textendash{} Spacing between values.

\item {} 
\sphinxstyleliteralstrong{\sphinxupquote{offset}} (\sphinxhref{https://docs.python.org/3/library/functions.html\#float}{\sphinxstyleliteralemphasis{\sphinxupquote{float}}}\sphinxstyleliteralemphasis{\sphinxupquote{ {[}}}\sphinxstyleliteralemphasis{\sphinxupquote{Hz}}\sphinxstyleliteralemphasis{\sphinxupquote{{]}}}) \textendash{} Offset frequency.

\item {} 
\sphinxstyleliteralstrong{\sphinxupquote{symetric}} (\sphinxhref{https://docs.python.org/3/library/functions.html\#bool}{\sphinxstyleliteralemphasis{\sphinxupquote{bool}}}) \textendash{} If true, generates the array between offset (-maximum shift,
maximum shift), otherwise from offset to maximum frequency. Default is True.

\item {} 
\sphinxstyleliteralstrong{\sphinxupquote{dtype}} (\sphinxstyleliteralemphasis{\sphinxupquote{np.dtype}}) \textendash{} Data type for the array. Default is np.float32.

\end{itemize}

\item[{Returns}] \leavevmode
Array of frequency shift value.

\item[{Return type}] \leavevmode
np.array {[}Hz{]}

\end{description}\end{quote}

\begin{sphinxadmonition}{note}{\label{autodoc/mrsprint/mrsprint.simulator:index-3}Todo:}
Use the key symetric for something.
\end{sphinxadmonition}

\end{fulllineitems}

\index{reduce\_magnetization\_in\_frequency() (in module mrsprint.simulator)}

\begin{fulllineitems}
\phantomsection\label{\detokenize{autodoc/mrsprint/mrsprint.simulator:mrsprint.simulator.reduce_magnetization_in_frequency}}\pysiglinewithargsret{\sphinxcode{\sphinxupquote{mrsprint.simulator.}}\sphinxbfcode{\sphinxupquote{reduce\_magnetization\_in\_frequency}}}{\emph{mx}, \emph{my}, \emph{mz}, \emph{freq\_shift}, \emph{fsa\_size}}{}
Reduces the magnetization vector by summing frequency components.
\begin{quote}\begin{description}
\item[{Parameters}] \leavevmode\begin{itemize}
\item {} 
\sphinxstyleliteralstrong{\sphinxupquote{mx}} (\sphinxstyleliteralemphasis{\sphinxupquote{np.array}}) \textendash{} Array of magnetization x without reduction in frequency.

\item {} 
\sphinxstyleliteralstrong{\sphinxupquote{my}} (\sphinxstyleliteralemphasis{\sphinxupquote{np.array}}) \textendash{} Array of magnetization y without reduction in frequency.

\item {} 
\sphinxstyleliteralstrong{\sphinxupquote{mz}} (\sphinxstyleliteralemphasis{\sphinxupquote{np.array}}) \textendash{} Array of magnetization z without reduction in frequency.

\item {} 
\sphinxstyleliteralstrong{\sphinxupquote{freq\_shift}} (\sphinxstyleliteralemphasis{\sphinxupquote{np.array}}) \textendash{} Array of frequency shift.

\item {} 
\sphinxstyleliteralstrong{\sphinxupquote{fsa\_size}} (\sphinxhref{https://docs.python.org/3/library/functions.html\#int}{\sphinxstyleliteralemphasis{\sphinxupquote{int}}}) \textendash{} Size of array of frequency shift.

\end{itemize}

\item[{Returns}] \leavevmode
(mx, my, mz) arrays of magnetization reduced

\item[{Return type}] \leavevmode
\sphinxhref{https://docs.python.org/3/library/stdtypes.html\#tuple}{tuple}

\end{description}\end{quote}

\begin{sphinxadmonition}{note}{\label{autodoc/mrsprint/mrsprint.simulator:index-4}Todo:}
Shape size different of 2.
Better solution for freq\_shift array.
Set physical unit for each args.
\end{sphinxadmonition}

\end{fulllineitems}

\index{reduce\_magnetization\_in\_position() (in module mrsprint.simulator)}

\begin{fulllineitems}
\phantomsection\label{\detokenize{autodoc/mrsprint/mrsprint.simulator:mrsprint.simulator.reduce_magnetization_in_position}}\pysiglinewithargsret{\sphinxcode{\sphinxupquote{mrsprint.simulator.}}\sphinxbfcode{\sphinxupquote{reduce\_magnetization\_in\_position}}}{\emph{mx}, \emph{my}, \emph{mz}, \emph{position}, \emph{freq\_shift}}{}
Reduces the magnetization vector by summing position components.
\begin{quote}\begin{description}
\item[{Parameters}] \leavevmode\begin{itemize}
\item {} 
\sphinxstyleliteralstrong{\sphinxupquote{mx}} (\sphinxstyleliteralemphasis{\sphinxupquote{np.array}}) \textendash{} Array of x magnetization without reduction in frequency.

\item {} 
\sphinxstyleliteralstrong{\sphinxupquote{my}} (\sphinxstyleliteralemphasis{\sphinxupquote{np.array}}) \textendash{} Array of y magnetization without reduction in frequency.

\item {} 
\sphinxstyleliteralstrong{\sphinxupquote{mz}} (\sphinxstyleliteralemphasis{\sphinxupquote{np.array}}) \textendash{} Array of z magnetization without reduction in frequency.

\item {} 
\sphinxstyleliteralstrong{\sphinxupquote{position}} (\sphinxstyleliteralemphasis{\sphinxupquote{np.array}}) \textendash{} Array of positions.

\item {} 
\sphinxstyleliteralstrong{\sphinxupquote{freq\_shift}} (\sphinxstyleliteralemphasis{\sphinxupquote{np.array}}) \textendash{} Array of frequency shift.

\end{itemize}

\item[{Returns}] \leavevmode
(mx, my, mz) arrays of magnetization reduced.

\item[{Return type}] \leavevmode
\sphinxhref{https://docs.python.org/3/library/stdtypes.html\#tuple}{tuple}

\end{description}\end{quote}

\begin{sphinxadmonition}{note}{\label{autodoc/mrsprint/mrsprint.simulator:index-5}Todo:}
Test shape size different of 2.
Set physical unit for each args.
\end{sphinxadmonition}

\end{fulllineitems}

\index{transform\_cart\_to\_pol() (in module mrsprint.simulator)}

\begin{fulllineitems}
\phantomsection\label{\detokenize{autodoc/mrsprint/mrsprint.simulator:mrsprint.simulator.transform_cart_to_pol}}\pysiglinewithargsret{\sphinxcode{\sphinxupquote{mrsprint.simulator.}}\sphinxbfcode{\sphinxupquote{transform\_cart\_to\_pol}}}{\emph{x}, \emph{y}}{}
Returns a transformed catesian vector into polar coordinates.
\begin{quote}\begin{description}
\item[{Parameters}] \leavevmode\begin{itemize}
\item {} 
\sphinxstyleliteralstrong{\sphinxupquote{x}} (\sphinxstyleliteralemphasis{\sphinxupquote{np.array}}) \textendash{} X coordinate.

\item {} 
\sphinxstyleliteralstrong{\sphinxupquote{y}} (\sphinxstyleliteralemphasis{\sphinxupquote{np.array}}) \textendash{} Y coordinate.

\end{itemize}

\item[{Returns}] \leavevmode
(rho, phi) vectors formed by radial and angular coordinates.

\item[{Return type}] \leavevmode
\sphinxhref{https://docs.python.org/3/library/stdtypes.html\#tuple}{tuple}

\end{description}\end{quote}

\end{fulllineitems}

\index{transform\_pol\_to\_cart() (in module mrsprint.simulator)}

\begin{fulllineitems}
\phantomsection\label{\detokenize{autodoc/mrsprint/mrsprint.simulator:mrsprint.simulator.transform_pol_to_cart}}\pysiglinewithargsret{\sphinxcode{\sphinxupquote{mrsprint.simulator.}}\sphinxbfcode{\sphinxupquote{transform\_pol\_to\_cart}}}{\emph{rho}, \emph{phi}}{}
Returns a transformed polar vector into cartesian coordinates.
\begin{quote}\begin{description}
\item[{Parameters}] \leavevmode\begin{itemize}
\item {} 
\sphinxstyleliteralstrong{\sphinxupquote{rho}} (\sphinxstyleliteralemphasis{\sphinxupquote{np.array}}) \textendash{} Radial coordinate.

\item {} 
\sphinxstyleliteralstrong{\sphinxupquote{phi}} (\sphinxstyleliteralemphasis{\sphinxupquote{np.array}}) \textendash{} Angular coordinate.

\end{itemize}

\item[{Returns}] \leavevmode
(x, y) vectors formed by x and y coordinates

\item[{Return type}] \leavevmode
\sphinxhref{https://docs.python.org/3/library/stdtypes.html\#tuple}{tuple}

\end{description}\end{quote}

\end{fulllineitems}



\paragraph{mrsprint.subject package}
\label{\detokenize{autodoc/mrsprint/mrsprint.subject:mrsprint-subject-package}}\label{\detokenize{autodoc/mrsprint/mrsprint.subject::doc}}

\subparagraph{Submodules}
\label{\detokenize{autodoc/mrsprint/mrsprint.subject:submodules}}

\subparagraph{mrsprint.subject.sample module}
\label{\detokenize{autodoc/mrsprint/mrsprint.subject:module-mrsprint.subject.sample}}\label{\detokenize{autodoc/mrsprint/mrsprint.subject:mrsprint-subject-sample-module}}\index{mrsprint.subject.sample (module)}
Module for sample related classes and functions.
\begin{description}
\item[{Authors:}] \leavevmode\begin{itemize}
\item {} 
Victor Hugo de Mello Pessoa \textless{}\sphinxhref{mailto:victor.pessoa@usp.br}{victor.pessoa@usp.br}\textgreater{}

\item {} 
Daniel Cosmo Pizetta \textless{}\sphinxhref{mailto:daniel.pizetta@usp.br}{daniel.pizetta@usp.br}\textgreater{}

\end{itemize}

\item[{Since:}] \leavevmode
2017/07/01

\end{description}
\index{Nucleus (class in mrsprint.subject.sample)}

\begin{fulllineitems}
\phantomsection\label{\detokenize{autodoc/mrsprint/mrsprint.subject:mrsprint.subject.sample.Nucleus}}\pysiglinewithargsret{\sphinxbfcode{\sphinxupquote{class }}\sphinxcode{\sphinxupquote{mrsprint.subject.sample.}}\sphinxbfcode{\sphinxupquote{Nucleus}}}{\emph{**opts}}{}
Bases: \sphinxcode{\sphinxupquote{pyqtgraph.parametertree.parameterTypes.GroupParameter}}

Class that represents the parameters in the nucleus.
\index{updateGamma() (mrsprint.subject.sample.Nucleus method)}

\begin{fulllineitems}
\phantomsection\label{\detokenize{autodoc/mrsprint/mrsprint.subject:mrsprint.subject.sample.Nucleus.updateGamma}}\pysiglinewithargsret{\sphinxbfcode{\sphinxupquote{updateGamma}}}{}{}
Update the value for gamma if the nucleus is changed

\end{fulllineitems}


\end{fulllineitems}

\index{Sample (class in mrsprint.subject.sample)}

\begin{fulllineitems}
\phantomsection\label{\detokenize{autodoc/mrsprint/mrsprint.subject:mrsprint.subject.sample.Sample}}\pysiglinewithargsret{\sphinxbfcode{\sphinxupquote{class }}\sphinxcode{\sphinxupquote{mrsprint.subject.sample.}}\sphinxbfcode{\sphinxupquote{Sample}}}{\emph{sample\_config}, \emph{**opts}}{}
Bases: \sphinxcode{\sphinxupquote{pyqtgraph.parametertree.parameterTypes.GroupParameter}}

Class that represents the parameters of each sample.
\begin{quote}\begin{description}
\item[{Parameters}] \leavevmode
\sphinxstyleliteralstrong{\sphinxupquote{sample\_config}} ({\hyperref[\detokenize{autodoc/mrsprint/mrsprint.subject:mrsprint.subject.sample.SampleConfig}]{\sphinxcrossref{\sphinxstyleliteralemphasis{\sphinxupquote{SampleConfig}}}}}) \textendash{} An object that represents the limits to this sample.

\end{description}\end{quote}

\begin{sphinxadmonition}{note}{\label{autodoc/mrsprint/mrsprint.subject:index-0}Todo:}
This class needs to be reviewed for its parameters.
\end{sphinxadmonition}
\index{dXUpdate() (mrsprint.subject.sample.Sample method)}

\begin{fulllineitems}
\phantomsection\label{\detokenize{autodoc/mrsprint/mrsprint.subject:mrsprint.subject.sample.Sample.dXUpdate}}\pysiglinewithargsret{\sphinxbfcode{\sphinxupquote{dXUpdate}}}{}{}
Update the value for dX.

\end{fulllineitems}

\index{dYUpdate() (mrsprint.subject.sample.Sample method)}

\begin{fulllineitems}
\phantomsection\label{\detokenize{autodoc/mrsprint/mrsprint.subject:mrsprint.subject.sample.Sample.dYUpdate}}\pysiglinewithargsret{\sphinxbfcode{\sphinxupquote{dYUpdate}}}{}{}
Update the value for dY.

\end{fulllineitems}

\index{dZUpdate() (mrsprint.subject.sample.Sample method)}

\begin{fulllineitems}
\phantomsection\label{\detokenize{autodoc/mrsprint/mrsprint.subject:mrsprint.subject.sample.Sample.dZUpdate}}\pysiglinewithargsret{\sphinxbfcode{\sphinxupquote{dZUpdate}}}{}{}
Update the value for dZ.

\end{fulllineitems}


\end{fulllineitems}

\index{SampleConfig (class in mrsprint.subject.sample)}

\begin{fulllineitems}
\phantomsection\label{\detokenize{autodoc/mrsprint/mrsprint.subject:mrsprint.subject.sample.SampleConfig}}\pysiglinewithargsret{\sphinxbfcode{\sphinxupquote{class }}\sphinxcode{\sphinxupquote{mrsprint.subject.sample.}}\sphinxbfcode{\sphinxupquote{SampleConfig}}}{\emph{**opts}}{}
Bases: \sphinxcode{\sphinxupquote{pyqtgraph.parametertree.parameterTypes.GroupParameter}}

Class that configure the limit parameters of each sample.

\begin{sphinxadmonition}{note}{\label{autodoc/mrsprint/mrsprint.subject:index-1}Todo:}
This class needs to be reviewed for its parameters.
\end{sphinxadmonition}

\end{fulllineitems}

\index{SampleElement (class in mrsprint.subject.sample)}

\begin{fulllineitems}
\phantomsection\label{\detokenize{autodoc/mrsprint/mrsprint.subject:mrsprint.subject.sample.SampleElement}}\pysiglinewithargsret{\sphinxbfcode{\sphinxupquote{class }}\sphinxcode{\sphinxupquote{mrsprint.subject.sample.}}\sphinxbfcode{\sphinxupquote{SampleElement}}}{\emph{sample\_element\_config}, \emph{**opts}}{}
Bases: \sphinxcode{\sphinxupquote{pyqtgraph.parametertree.parameterTypes.GroupParameter}}

Class that represents the parameters of each sample element.
\begin{quote}\begin{description}
\item[{Parameters}] \leavevmode
\sphinxstyleliteralstrong{\sphinxupquote{sample\_element\_config}} ({\hyperref[\detokenize{autodoc/mrsprint/mrsprint.subject:mrsprint.subject.sample.SampleElementConfig}]{\sphinxcrossref{\sphinxstyleliteralemphasis{\sphinxupquote{SampleElementConfig}}}}}) \textendash{} An object that represents the limits to the sample elements.

\end{description}\end{quote}

\begin{sphinxadmonition}{note}{\label{autodoc/mrsprint/mrsprint.subject:index-2}Todo:}
This class needs to be reviewed for its parameters.
\end{sphinxadmonition}

\end{fulllineitems}

\index{SampleElementConfig (class in mrsprint.subject.sample)}

\begin{fulllineitems}
\phantomsection\label{\detokenize{autodoc/mrsprint/mrsprint.subject:mrsprint.subject.sample.SampleElementConfig}}\pysiglinewithargsret{\sphinxbfcode{\sphinxupquote{class }}\sphinxcode{\sphinxupquote{mrsprint.subject.sample.}}\sphinxbfcode{\sphinxupquote{SampleElementConfig}}}{\emph{**opts}}{}
Bases: \sphinxcode{\sphinxupquote{pyqtgraph.parametertree.parameterTypes.GroupParameter}}

Class that configure the limit parameters of each sample element.

\begin{sphinxadmonition}{note}{\label{autodoc/mrsprint/mrsprint.subject:index-3}Todo:}
This class needs to be reviewed for its parameters.
\end{sphinxadmonition}

\end{fulllineitems}



\subparagraph{Module contents}
\label{\detokenize{autodoc/mrsprint/mrsprint.subject:module-mrsprint.subject}}\label{\detokenize{autodoc/mrsprint/mrsprint.subject:module-contents}}\index{mrsprint.subject (module)}
Package for subject related classes and objects.
\begin{description}
\item[{Authors:}] \leavevmode\begin{itemize}
\item {} 
Daniel Cosmo Pizetta \textless{}\sphinxhref{mailto:daniel.pizetta@usp.br}{daniel.pizetta@usp.br}\textgreater{}

\end{itemize}

\item[{Since:}] \leavevmode
2017/10/01

\end{description}


\paragraph{mrsprint.system package}
\label{\detokenize{autodoc/mrsprint/mrsprint.system:mrsprint-system-package}}\label{\detokenize{autodoc/mrsprint/mrsprint.system::doc}}

\subparagraph{Submodules}
\label{\detokenize{autodoc/mrsprint/mrsprint.system:submodules}}

\subparagraph{mrsprint.system.gradient module}
\label{\detokenize{autodoc/mrsprint/mrsprint.system:module-mrsprint.system.gradient}}\label{\detokenize{autodoc/mrsprint/mrsprint.system:mrsprint-system-gradient-module}}\index{mrsprint.system.gradient (module)}
Module for gradient related classes and functions.
\begin{description}
\item[{Authors:}] \leavevmode\begin{itemize}
\item {} 
Victor Hugo de Mello Pessoa \textless{}\sphinxhref{mailto:victor.pessoa@usp.br}{victor.pessoa@usp.br}\textgreater{}

\item {} 
Daniel Cosmo Pizetta \textless{}\sphinxhref{mailto:daniel.pizetta@usp.br}{daniel.pizetta@usp.br}\textgreater{}

\end{itemize}

\item[{Since:}] \leavevmode
2017/07/01

\end{description}

\begin{sphinxadmonition}{note}{\label{autodoc/mrsprint/mrsprint.system:index-0}Todo:}
Change nameOfFunctions to name\_of\_functions.
\end{sphinxadmonition}
\index{Gradient (class in mrsprint.system.gradient)}

\begin{fulllineitems}
\phantomsection\label{\detokenize{autodoc/mrsprint/mrsprint.system:mrsprint.system.gradient.Gradient}}\pysiglinewithargsret{\sphinxbfcode{\sphinxupquote{class }}\sphinxcode{\sphinxupquote{mrsprint.system.gradient.}}\sphinxbfcode{\sphinxupquote{Gradient}}}{\emph{**opts}}{}
Bases: \sphinxcode{\sphinxupquote{pyqtgraph.parametertree.parameterTypes.GroupParameter}}

Class that represents the gradients parameters in the sistem.

\end{fulllineitems}

\index{gradient\_delay() (in module mrsprint.system.gradient)}

\begin{fulllineitems}
\phantomsection\label{\detokenize{autodoc/mrsprint/mrsprint.system:mrsprint.system.gradient.gradient_delay}}\pysiglinewithargsret{\sphinxcode{\sphinxupquote{mrsprint.system.gradient.}}\sphinxbfcode{\sphinxupquote{gradient\_delay}}}{\emph{duration}, \emph{dt}, \emph{number\_of\_points=0}}{}
Generate a delay of gradient pulse.
\begin{quote}\begin{description}
\item[{Parameters}] \leavevmode\begin{itemize}
\item {} 
\sphinxstyleliteralstrong{\sphinxupquote{duration}} (\sphinxhref{https://docs.python.org/3/library/functions.html\#float}{\sphinxstyleliteralemphasis{\sphinxupquote{float}}}\sphinxstyleliteralemphasis{\sphinxupquote{ {[}}}\sphinxstyleliteralemphasis{\sphinxupquote{s}}\sphinxstyleliteralemphasis{\sphinxupquote{{]}}}) \textendash{} Delay time.

\item {} 
\sphinxstyleliteralstrong{\sphinxupquote{dt}} (\sphinxhref{https://docs.python.org/3/library/functions.html\#float}{\sphinxstyleliteralemphasis{\sphinxupquote{float}}}\sphinxstyleliteralemphasis{\sphinxupquote{ {[}}}\sphinxstyleliteralemphasis{\sphinxupquote{s}}\sphinxstyleliteralemphasis{\sphinxupquote{{]}}}) \textendash{} Time resolution.

\item {} 
\sphinxstyleliteralstrong{\sphinxupquote{number\_of\_points}} (\sphinxhref{https://docs.python.org/3/library/functions.html\#int}{\sphinxstyleliteralemphasis{\sphinxupquote{int}}}) \textendash{} Number of points.

\end{itemize}

\item[{Returns}] \leavevmode
Gradient delay - a zero x, y and z gradient components

\item[{Return type}] \leavevmode
np.array

\end{description}\end{quote}

\end{fulllineitems}

\index{gradient\_duration() (in module mrsprint.system.gradient)}

\begin{fulllineitems}
\phantomsection\label{\detokenize{autodoc/mrsprint/mrsprint.system:mrsprint.system.gradient.gradient_duration}}\pysiglinewithargsret{\sphinxcode{\sphinxupquote{mrsprint.system.gradient.}}\sphinxbfcode{\sphinxupquote{gradient\_duration}}}{\emph{gradient\_event}, \emph{dt}}{}
Return the duration of the gradient event (array).

It is based on the size and dt.
\begin{quote}\begin{description}
\item[{Parameters}] \leavevmode\begin{itemize}
\item {} 
\sphinxstyleliteralstrong{\sphinxupquote{gradient\_event}} (\sphinxstyleliteralemphasis{\sphinxupquote{np.array}}) \textendash{} An array of event.

\item {} 
\sphinxstyleliteralstrong{\sphinxupquote{dt}} (\sphinxhref{https://docs.python.org/3/library/functions.html\#float}{\sphinxstyleliteralemphasis{\sphinxupquote{float}}}\sphinxstyleliteralemphasis{\sphinxupquote{ {[}}}\sphinxstyleliteralemphasis{\sphinxupquote{s}}\sphinxstyleliteralemphasis{\sphinxupquote{{]}}}) \textendash{} Value of time resolution.

\end{itemize}

\item[{Returns}] \leavevmode
Duration of the event.

\item[{Return type}] \leavevmode
\sphinxhref{https://docs.python.org/3/library/functions.html\#float}{float} {[}s{]}

\end{description}\end{quote}

\begin{sphinxadmonition}{note}{\label{autodoc/mrsprint/mrsprint.system:index-1}Todo:}
Regard the dimensions of the array.
\end{sphinxadmonition}

\end{fulllineitems}



\subparagraph{mrsprint.system.magnet module}
\label{\detokenize{autodoc/mrsprint/mrsprint.system:module-mrsprint.system.magnet}}\label{\detokenize{autodoc/mrsprint/mrsprint.system:mrsprint-system-magnet-module}}\index{mrsprint.system.magnet (module)}
Module for magnet related classes and functions.
\begin{description}
\item[{Authors:}] \leavevmode\begin{itemize}
\item {} 
Victor Hugo de Mello Pessoa \textless{}\sphinxhref{mailto:victor.pessoa@usp.br}{victor.pessoa@usp.br}\textgreater{}

\item {} 
Daniel Cosmo Pizetta \textless{}\sphinxhref{mailto:daniel.pizetta@usp.br}{daniel.pizetta@usp.br}\textgreater{}

\end{itemize}

\item[{Since:}] \leavevmode
2017/07/01

\end{description}

\begin{sphinxadmonition}{note}{\label{autodoc/mrsprint/mrsprint.system:index-2}Todo:}
Maybe put all config together in magnet classes.
\end{sphinxadmonition}
\index{Magnet (class in mrsprint.system.magnet)}

\begin{fulllineitems}
\phantomsection\label{\detokenize{autodoc/mrsprint/mrsprint.system:mrsprint.system.magnet.Magnet}}\pysiglinewithargsret{\sphinxbfcode{\sphinxupquote{class }}\sphinxcode{\sphinxupquote{mrsprint.system.magnet.}}\sphinxbfcode{\sphinxupquote{Magnet}}}{\emph{magnet\_config}, \emph{**opts}}{}
Bases: \sphinxcode{\sphinxupquote{pyqtgraph.parametertree.parameterTypes.GroupParameter}}

Class that represents the parameters in the magnet.
\begin{quote}\begin{description}
\item[{Parameters}] \leavevmode
\sphinxstyleliteralstrong{\sphinxupquote{magnet\_config}} ({\hyperref[\detokenize{autodoc/mrsprint/mrsprint.system:mrsprint.system.magnet.MagnetConfig}]{\sphinxcrossref{\sphinxstyleliteralemphasis{\sphinxupquote{MagnetConfig}}}}}) \textendash{} An object that represents the limits to this magnet.

\end{description}\end{quote}

\end{fulllineitems}

\index{MagnetConfig (class in mrsprint.system.magnet)}

\begin{fulllineitems}
\phantomsection\label{\detokenize{autodoc/mrsprint/mrsprint.system:mrsprint.system.magnet.MagnetConfig}}\pysiglinewithargsret{\sphinxbfcode{\sphinxupquote{class }}\sphinxcode{\sphinxupquote{mrsprint.system.magnet.}}\sphinxbfcode{\sphinxupquote{MagnetConfig}}}{\emph{**opts}}{}
Bases: \sphinxcode{\sphinxupquote{pyqtgraph.parametertree.parameterTypes.GroupParameter}}

Class that configure the limit parameters of the magnet.

\end{fulllineitems}



\subparagraph{mrsprint.system.rf module}
\label{\detokenize{autodoc/mrsprint/mrsprint.system:module-mrsprint.system.rf}}\label{\detokenize{autodoc/mrsprint/mrsprint.system:mrsprint-system-rf-module}}\index{mrsprint.system.rf (module)}
Module for radiofrequency related classes and functions.
\begin{description}
\item[{Authors:}] \leavevmode\begin{itemize}
\item {} 
Victor Hugo de Mello Pessoa \textless{}\sphinxhref{mailto:victor.pessoa@usp.br}{victor.pessoa@usp.br}\textgreater{}

\item {} 
Daniel Cosmo Pizetta \textless{}\sphinxhref{mailto:daniel.pizetta@usp.br}{daniel.pizetta@usp.br}\textgreater{}

\end{itemize}

\item[{Since:}] \leavevmode
2017/07/01

\end{description}
\index{RF (class in mrsprint.system.rf)}

\begin{fulllineitems}
\phantomsection\label{\detokenize{autodoc/mrsprint/mrsprint.system:mrsprint.system.rf.RF}}\pysiglinewithargsret{\sphinxbfcode{\sphinxupquote{class }}\sphinxcode{\sphinxupquote{mrsprint.system.rf.}}\sphinxbfcode{\sphinxupquote{RF}}}{\emph{**opts}}{}
Bases: \sphinxcode{\sphinxupquote{pyqtgraph.parametertree.parameterTypes.GroupParameter}}

Class that represents the RF parameters in the system.

\end{fulllineitems}

\index{rf\_delay() (in module mrsprint.system.rf)}

\begin{fulllineitems}
\phantomsection\label{\detokenize{autodoc/mrsprint/mrsprint.system:mrsprint.system.rf.rf_delay}}\pysiglinewithargsret{\sphinxcode{\sphinxupquote{mrsprint.system.rf.}}\sphinxbfcode{\sphinxupquote{rf\_delay}}}{\emph{duration}, \emph{dt}}{}
Generate a delay of rf pulse.
\begin{quote}\begin{description}
\item[{Parameters}] \leavevmode\begin{itemize}
\item {} 
\sphinxstyleliteralstrong{\sphinxupquote{duration}} (\sphinxhref{https://docs.python.org/3/library/functions.html\#float}{\sphinxstyleliteralemphasis{\sphinxupquote{float}}}\sphinxstyleliteralemphasis{\sphinxupquote{ {[}}}\sphinxstyleliteralemphasis{\sphinxupquote{s}}\sphinxstyleliteralemphasis{\sphinxupquote{{]}}}) \textendash{} Delay time.

\item {} 
\sphinxstyleliteralstrong{\sphinxupquote{dt}} (\sphinxhref{https://docs.python.org/3/library/functions.html\#float}{\sphinxstyleliteralemphasis{\sphinxupquote{float}}}\sphinxstyleliteralemphasis{\sphinxupquote{ {[}}}\sphinxstyleliteralemphasis{\sphinxupquote{s}}\sphinxstyleliteralemphasis{\sphinxupquote{{]}}}) \textendash{} Time resolution.

\end{itemize}

\item[{Returns}] \leavevmode
Rf delay - a zero am, pm, fm components in complex format.

\item[{Return type}] \leavevmode
np.array(\sphinxhref{https://docs.python.org/3/library/functions.html\#complex}{complex})

\end{description}\end{quote}

\end{fulllineitems}

\index{rf\_duration() (in module mrsprint.system.rf)}

\begin{fulllineitems}
\phantomsection\label{\detokenize{autodoc/mrsprint/mrsprint.system:mrsprint.system.rf.rf_duration}}\pysiglinewithargsret{\sphinxcode{\sphinxupquote{mrsprint.system.rf.}}\sphinxbfcode{\sphinxupquote{rf\_duration}}}{\emph{rf\_event}, \emph{dt}}{}
Return the duration of the rf event (array) based on the number of the points and  dt.
\begin{quote}\begin{description}
\item[{Parameters}] \leavevmode\begin{itemize}
\item {} 
\sphinxstyleliteralstrong{\sphinxupquote{rf\_event}} (\sphinxstyleliteralemphasis{\sphinxupquote{np.array}}) \textendash{} An array of event.

\item {} 
\sphinxstyleliteralstrong{\sphinxupquote{dt}} (\sphinxhref{https://docs.python.org/3/library/functions.html\#float}{\sphinxstyleliteralemphasis{\sphinxupquote{float}}}\sphinxstyleliteralemphasis{\sphinxupquote{ {[}}}\sphinxstyleliteralemphasis{\sphinxupquote{s}}\sphinxstyleliteralemphasis{\sphinxupquote{{]}}}) \textendash{} Value of time resolution.

\end{itemize}

\item[{Returns}] \leavevmode
duration of the event.

\item[{Return type}] \leavevmode
\sphinxhref{https://docs.python.org/3/library/functions.html\#float}{float} {[}s{]}

\end{description}\end{quote}

\begin{sphinxadmonition}{note}{\label{autodoc/mrsprint/mrsprint.system:index-3}Todo:}
Regard the dimensions of the array.
\end{sphinxadmonition}

\end{fulllineitems}

\index{square\_rf\_pulse() (in module mrsprint.system.rf)}

\begin{fulllineitems}
\phantomsection\label{\detokenize{autodoc/mrsprint/mrsprint.system:mrsprint.system.rf.square_rf_pulse}}\pysiglinewithargsret{\sphinxcode{\sphinxupquote{mrsprint.system.rf.}}\sphinxbfcode{\sphinxupquote{square\_rf\_pulse}}}{\emph{dt}, \emph{gamma}, \emph{b1\_max}, \emph{flip\_angle=90}, \emph{phase=0.0}, \emph{degrees=True}}{}
Generate a hard rf pulse with a specific and constant flip angle and phase.
\begin{quote}\begin{description}
\item[{Parameters}] \leavevmode\begin{itemize}
\item {} 
\sphinxstyleliteralstrong{\sphinxupquote{dt}} (\sphinxhref{https://docs.python.org/3/library/functions.html\#float}{\sphinxstyleliteralemphasis{\sphinxupquote{float}}}\sphinxstyleliteralemphasis{\sphinxupquote{ {[}}}\sphinxstyleliteralemphasis{\sphinxupquote{s}}\sphinxstyleliteralemphasis{\sphinxupquote{{]}}}) \textendash{} Value of time resolution.

\item {} 
\sphinxstyleliteralstrong{\sphinxupquote{gamma}} (\sphinxhref{https://docs.python.org/3/library/functions.html\#float}{\sphinxstyleliteralemphasis{\sphinxupquote{float}}}\sphinxstyleliteralemphasis{\sphinxupquote{ {[}}}\sphinxstyleliteralemphasis{\sphinxupquote{rad/}}\sphinxstyleliteralemphasis{\sphinxupquote{(}}\sphinxstyleliteralemphasis{\sphinxupquote{G*s}}\sphinxstyleliteralemphasis{\sphinxupquote{)}}\sphinxstyleliteralemphasis{\sphinxupquote{{]}}}) \textendash{} Gyromagnetic ratio of the excited nuclei.

\item {} 
\sphinxstyleliteralstrong{\sphinxupquote{b1\_max}} (\sphinxhref{https://docs.python.org/3/library/functions.html\#float}{\sphinxstyleliteralemphasis{\sphinxupquote{float}}}\sphinxstyleliteralemphasis{\sphinxupquote{ {[}}}\sphinxstyleliteralemphasis{\sphinxupquote{G}}\sphinxstyleliteralemphasis{\sphinxupquote{{]}}}) \textendash{} Max RF amplitude.

\item {} 
\sphinxstyleliteralstrong{\sphinxupquote{flip\_angle}} (\sphinxhref{https://docs.python.org/3/library/functions.html\#float}{\sphinxstyleliteralemphasis{\sphinxupquote{float}}}\sphinxstyleliteralemphasis{\sphinxupquote{ {[}}}\sphinxstyleliteralemphasis{\sphinxupquote{degrees}}\sphinxstyleliteralemphasis{\sphinxupquote{, }}\sphinxstyleliteralemphasis{\sphinxupquote{radians}}\sphinxstyleliteralemphasis{\sphinxupquote{{]}}}) \textendash{} Array of flip angle for rf pulse in degrees (if degrees = True).

\item {} 
\sphinxstyleliteralstrong{\sphinxupquote{phase}} (\sphinxhref{https://docs.python.org/3/library/functions.html\#float}{\sphinxstyleliteralemphasis{\sphinxupquote{float}}}\sphinxstyleliteralemphasis{\sphinxupquote{ {[}}}\sphinxstyleliteralemphasis{\sphinxupquote{degrees}}\sphinxstyleliteralemphasis{\sphinxupquote{, }}\sphinxstyleliteralemphasis{\sphinxupquote{radians}}\sphinxstyleliteralemphasis{\sphinxupquote{{]}}}) \textendash{} Array of phase angle for rf pulse in degrees (if degrees = True).

\item {} 
\sphinxstyleliteralstrong{\sphinxupquote{degrees}} (\sphinxhref{https://docs.python.org/3/library/functions.html\#bool}{\sphinxstyleliteralemphasis{\sphinxupquote{bool}}}) \textendash{} Inform if the input is in degrees or radians.

\end{itemize}

\item[{Returns}] \leavevmode
A square rf pulse in imaginary form.

\item[{Return type}] \leavevmode
np.array(\sphinxhref{https://docs.python.org/3/library/functions.html\#complex}{complex})

\end{description}\end{quote}

\end{fulllineitems}



\subparagraph{Module contents}
\label{\detokenize{autodoc/mrsprint/mrsprint.system:module-mrsprint.system}}\label{\detokenize{autodoc/mrsprint/mrsprint.system:module-contents}}\index{mrsprint.system (module)}
Package for system related classes and objects.
\begin{description}
\item[{Authors:}] \leavevmode\begin{itemize}
\item {} 
Daniel Cosmo Pizetta \textless{}\sphinxhref{mailto:daniel.pizetta@usp.br}{daniel.pizetta@usp.br}\textgreater{}

\end{itemize}

\item[{Since:}] \leavevmode
2015/11/01

\end{description}


\subsubsection{Submodules}
\label{\detokenize{autodoc/mrsprint/mrsprint:submodules}}

\subsubsection{mrsprint.globals module}
\label{\detokenize{autodoc/mrsprint/mrsprint:module-mrsprint.globals}}\label{\detokenize{autodoc/mrsprint/mrsprint:mrsprint-globals-module}}\index{mrsprint.globals (module)}
Global values.
\begin{description}
\item[{Authors:}] \leavevmode\begin{itemize}
\item {} 
Victor Hugo de Mello Pessoa \textless{}\sphinxhref{mailto:victor.pessoa@usp.br}{victor.pessoa@usp.br}\textgreater{}

\item {} 
Daniel Cosmo Pizetta \textless{}\sphinxhref{mailto:daniel.pizetta@usp.br}{daniel.pizetta@usp.br}\textgreater{}

\end{itemize}

\item[{Since:}] \leavevmode
2015/06/01

\end{description}


\subsubsection{mrsprint.mainwindow module}
\label{\detokenize{autodoc/mrsprint/mrsprint:module-mrsprint.mainwindow}}\label{\detokenize{autodoc/mrsprint/mrsprint:mrsprint-mainwindow-module}}\index{mrsprint.mainwindow (module)}
Main window of visual simulator.
\begin{description}
\item[{Authors:}] \leavevmode\begin{itemize}
\item {} 
Victor Hugo de Mello Pessoa \textless{}\sphinxhref{mailto:victor.pessoa@usp.br}{victor.pessoa@usp.br}\textgreater{}

\item {} 
Daniel Cosmo Pizetta \textless{}\sphinxhref{mailto:daniel.pizetta@usp.br}{daniel.pizetta@usp.br}\textgreater{}

\end{itemize}

\item[{Since:}] \leavevmode
2017/08/01

\end{description}

\begin{sphinxadmonition}{note}{\label{autodoc/mrsprint/mrsprint:index-0}Todo:}
Replaning this module, divide and make it more simple.
\end{sphinxadmonition}
\index{MainWindow (class in mrsprint.mainwindow)}

\begin{fulllineitems}
\phantomsection\label{\detokenize{autodoc/mrsprint/mrsprint:mrsprint.mainwindow.MainWindow}}\pysiglinewithargsret{\sphinxbfcode{\sphinxupquote{class }}\sphinxcode{\sphinxupquote{mrsprint.mainwindow.}}\sphinxbfcode{\sphinxupquote{MainWindow}}}{\emph{parent=None}}{}
Bases: \sphinxcode{\sphinxupquote{PyQt4.QtGui.QMainWindow}}

Main window.
\begin{description}
\item[{About load methods:}] \leavevmode
This method should exist for each type of context that could be edited
in the 2D editor and 3D view. Ex. loadSystemParameters. It should be
responsible for loading values from parameter tree, connect signals -
to respective own actions, and set data.To keep it isolated, it should
set data, and when data is changed, data triggers the GUI update.
And the reverse mode.

\end{description}
\index{about() (mrsprint.mainwindow.MainWindow method)}

\begin{fulllineitems}
\phantomsection\label{\detokenize{autodoc/mrsprint/mrsprint:mrsprint.mainwindow.MainWindow.about}}\pysiglinewithargsret{\sphinxbfcode{\sphinxupquote{about}}}{}{}
Show about message.

\end{fulllineitems}

\index{canClose() (mrsprint.mainwindow.MainWindow method)}

\begin{fulllineitems}
\phantomsection\label{\detokenize{autodoc/mrsprint/mrsprint:mrsprint.mainwindow.MainWindow.canClose}}\pysiglinewithargsret{\sphinxbfcode{\sphinxupquote{canClose}}}{}{}
Check if the user want to save the sample before closing.
\begin{quote}\begin{description}
\item[{Returns}] \leavevmode
True if can close.

\item[{Return type}] \leavevmode
\sphinxhref{https://docs.python.org/3/library/functions.html\#bool}{bool}

\end{description}\end{quote}

\end{fulllineitems}

\index{clearSelection2DEditor() (mrsprint.mainwindow.MainWindow method)}

\begin{fulllineitems}
\phantomsection\label{\detokenize{autodoc/mrsprint/mrsprint:mrsprint.mainwindow.MainWindow.clearSelection2DEditor}}\pysiglinewithargsret{\sphinxbfcode{\sphinxupquote{clearSelection2DEditor}}}{\emph{all\_tables=False}}{}
Clear selection of all 2D editor tables.
\begin{quote}\begin{description}
\item[{Parameters}] \leavevmode
\sphinxstyleliteralstrong{\sphinxupquote{all\_tables}} (\sphinxhref{https://docs.python.org/3/library/functions.html\#bool}{\sphinxstyleliteralemphasis{\sphinxupquote{bool}}}) \textendash{} Informs if all tables are shown or not. Default is False.

\end{description}\end{quote}

\end{fulllineitems}

\index{closeEvent() (mrsprint.mainwindow.MainWindow method)}

\begin{fulllineitems}
\phantomsection\label{\detokenize{autodoc/mrsprint/mrsprint:mrsprint.mainwindow.MainWindow.closeEvent}}\pysiglinewithargsret{\sphinxbfcode{\sphinxupquote{closeEvent}}}{\emph{event}}{}
Close event.

\end{fulllineitems}

\index{create3DView() (mrsprint.mainwindow.MainWindow method)}

\begin{fulllineitems}
\phantomsection\label{\detokenize{autodoc/mrsprint/mrsprint:mrsprint.mainwindow.MainWindow.create3DView}}\pysiglinewithargsret{\sphinxbfcode{\sphinxupquote{create3DView}}}{}{}
Create a 3D view and the first 3D object.

It must be called when starts and when the shape or size is changed.

\end{fulllineitems}

\index{enableGradient2DEditor() (mrsprint.mainwindow.MainWindow method)}

\begin{fulllineitems}
\phantomsection\label{\detokenize{autodoc/mrsprint/mrsprint:mrsprint.mainwindow.MainWindow.enableGradient2DEditor}}\pysiglinewithargsret{\sphinxbfcode{\sphinxupquote{enableGradient2DEditor}}}{}{}
Enable the gradient editor for 2D editor.

\end{fulllineitems}

\index{fileClose() (mrsprint.mainwindow.MainWindow method)}

\begin{fulllineitems}
\phantomsection\label{\detokenize{autodoc/mrsprint/mrsprint:mrsprint.mainwindow.MainWindow.fileClose}}\pysiglinewithargsret{\sphinxbfcode{\sphinxupquote{fileClose}}}{}{}
Close file and create a new one.

\end{fulllineitems}

\index{fileNew() (mrsprint.mainwindow.MainWindow method)}

\begin{fulllineitems}
\phantomsection\label{\detokenize{autodoc/mrsprint/mrsprint:mrsprint.mainwindow.MainWindow.fileNew}}\pysiglinewithargsret{\sphinxbfcode{\sphinxupquote{fileNew}}}{}{}
Restart the edition from start.

\end{fulllineitems}

\index{fileOpen() (mrsprint.mainwindow.MainWindow method)}

\begin{fulllineitems}
\phantomsection\label{\detokenize{autodoc/mrsprint/mrsprint:mrsprint.mainwindow.MainWindow.fileOpen}}\pysiglinewithargsret{\sphinxbfcode{\sphinxupquote{fileOpen}}}{}{}
Open a dialog to select a HDF5 file to be opened.

\end{fulllineitems}

\index{fileSave() (mrsprint.mainwindow.MainWindow method)}

\begin{fulllineitems}
\phantomsection\label{\detokenize{autodoc/mrsprint/mrsprint:mrsprint.mainwindow.MainWindow.fileSave}}\pysiglinewithargsret{\sphinxbfcode{\sphinxupquote{fileSave}}}{}{}
Save the current sample in the last saving file.

If there is no current saving file, it calls fileSaveAs().

\end{fulllineitems}

\index{fileSaveAs() (mrsprint.mainwindow.MainWindow method)}

\begin{fulllineitems}
\phantomsection\label{\detokenize{autodoc/mrsprint/mrsprint:mrsprint.mainwindow.MainWindow.fileSaveAs}}\pysiglinewithargsret{\sphinxbfcode{\sphinxupquote{fileSaveAs}}}{}{}
Open a dialog to select the current saving file.

\end{fulllineitems}

\index{gradient2DEditor() (mrsprint.mainwindow.MainWindow method)}

\begin{fulllineitems}
\phantomsection\label{\detokenize{autodoc/mrsprint/mrsprint:mrsprint.mainwindow.MainWindow.gradient2DEditor}}\pysiglinewithargsret{\sphinxbfcode{\sphinxupquote{gradient2DEditor}}}{}{}
Run the gradient editor and set values to 2D editor tables.

\begin{sphinxadmonition}{note}{\label{autodoc/mrsprint/mrsprint:index-1}Todo:}
Remove eval after changing itemMethodsTable2dEditor.
\end{sphinxadmonition}

\end{fulllineitems}

\index{invertBackground2DEditor() (mrsprint.mainwindow.MainWindow method)}

\begin{fulllineitems}
\phantomsection\label{\detokenize{autodoc/mrsprint/mrsprint:mrsprint.mainwindow.MainWindow.invertBackground2DEditor}}\pysiglinewithargsret{\sphinxbfcode{\sphinxupquote{invertBackground2DEditor}}}{\emph{invert=False}}{}
Invert the background color of 2D editor.

\end{fulllineitems}

\index{invertBackground3DView() (mrsprint.mainwindow.MainWindow method)}

\begin{fulllineitems}
\phantomsection\label{\detokenize{autodoc/mrsprint/mrsprint:mrsprint.mainwindow.MainWindow.invertBackground3DView}}\pysiglinewithargsret{\sphinxbfcode{\sphinxupquote{invertBackground3DView}}}{\emph{invert=False}}{}
Invert the background color of 3D view.
\begin{quote}\begin{description}
\item[{Parameters}] \leavevmode
\sphinxstyleliteralstrong{\sphinxupquote{invert}} (\sphinxhref{https://docs.python.org/3/library/functions.html\#bool}{\sphinxstyleliteralemphasis{\sphinxupquote{bool}}}) \textendash{} Informs if the background color should be inverted. Default is False.

\end{description}\end{quote}

\end{fulllineitems}

\index{itemsMethodsTable2DEditor() (mrsprint.mainwindow.MainWindow method)}

\begin{fulllineitems}
\phantomsection\label{\detokenize{autodoc/mrsprint/mrsprint:mrsprint.mainwindow.MainWindow.itemsMethodsTable2DEditor}}\pysiglinewithargsret{\sphinxbfcode{\sphinxupquote{itemsMethodsTable2DEditor}}}{}{}
Return current selected item, methods to access their indexes and the table.
\begin{quote}\begin{description}
\item[{Returns}] \leavevmode
The current selected items, the methods attached to the selected table and the selected table

\item[{Return type}] \leavevmode
\sphinxhref{https://docs.python.org/3/library/stdtypes.html\#list}{list} (QtGui.QTableWidgetItem()), \sphinxhref{https://docs.python.org/3/library/stdtypes.html\#dict}{dict} (string of methods), QtGui.QTableWidget()

\end{description}\end{quote}

\begin{sphinxadmonition}{note}{\label{autodoc/mrsprint/mrsprint:index-2}Todo:}
This could be even better if getting these values, return just a new matrix
with proper indexes and values to set data.
\end{sphinxadmonition}

\end{fulllineitems}

\index{loadProcessingParameters() (mrsprint.mainwindow.MainWindow method)}

\begin{fulllineitems}
\phantomsection\label{\detokenize{autodoc/mrsprint/mrsprint:mrsprint.mainwindow.MainWindow.loadProcessingParameters}}\pysiglinewithargsret{\sphinxbfcode{\sphinxupquote{loadProcessingParameters}}}{}{}
Load processing parameters and connect signals.

\end{fulllineitems}

\index{loadSampleParameters() (mrsprint.mainwindow.MainWindow method)}

\begin{fulllineitems}
\phantomsection\label{\detokenize{autodoc/mrsprint/mrsprint:mrsprint.mainwindow.MainWindow.loadSampleParameters}}\pysiglinewithargsret{\sphinxbfcode{\sphinxupquote{loadSampleParameters}}}{}{}
Load sample context and its parameters.

\end{fulllineitems}

\index{loadSequenceParameters() (mrsprint.mainwindow.MainWindow method)}

\begin{fulllineitems}
\phantomsection\label{\detokenize{autodoc/mrsprint/mrsprint:mrsprint.mainwindow.MainWindow.loadSequenceParameters}}\pysiglinewithargsret{\sphinxbfcode{\sphinxupquote{loadSequenceParameters}}}{}{}
Load sequence context and its parameters.

\end{fulllineitems}

\index{loadSettings() (mrsprint.mainwindow.MainWindow method)}

\begin{fulllineitems}
\phantomsection\label{\detokenize{autodoc/mrsprint/mrsprint:mrsprint.mainwindow.MainWindow.loadSettings}}\pysiglinewithargsret{\sphinxbfcode{\sphinxupquote{loadSettings}}}{}{}
Load all the config parameters needed by the software.

\end{fulllineitems}

\index{loadSimulatorParameters() (mrsprint.mainwindow.MainWindow method)}

\begin{fulllineitems}
\phantomsection\label{\detokenize{autodoc/mrsprint/mrsprint:mrsprint.mainwindow.MainWindow.loadSimulatorParameters}}\pysiglinewithargsret{\sphinxbfcode{\sphinxupquote{loadSimulatorParameters}}}{}{}
Load simulator context and its parameters.

\end{fulllineitems}

\index{loadSystemParameters() (mrsprint.mainwindow.MainWindow method)}

\begin{fulllineitems}
\phantomsection\label{\detokenize{autodoc/mrsprint/mrsprint:mrsprint.mainwindow.MainWindow.loadSystemParameters}}\pysiglinewithargsret{\sphinxbfcode{\sphinxupquote{loadSystemParameters}}}{}{}
Load system context and its parameters.

\end{fulllineitems}

\index{open() (mrsprint.mainwindow.MainWindow method)}

\begin{fulllineitems}
\phantomsection\label{\detokenize{autodoc/mrsprint/mrsprint:mrsprint.mainwindow.MainWindow.open}}\pysiglinewithargsret{\sphinxbfcode{\sphinxupquote{open}}}{\emph{file\_path}}{}
Open a HDF5 file and applies the changes to the program.
\begin{quote}\begin{description}
\item[{Parameters}] \leavevmode
\sphinxstyleliteralstrong{\sphinxupquote{file\_path}} (\sphinxhref{https://docs.python.org/3/library/stdtypes.html\#str}{\sphinxstyleliteralemphasis{\sphinxupquote{str}}}) \textendash{} Path to the file where the data will be opened.

\end{description}\end{quote}

\end{fulllineitems}

\index{openSample() (mrsprint.mainwindow.MainWindow method)}

\begin{fulllineitems}
\phantomsection\label{\detokenize{autodoc/mrsprint/mrsprint:mrsprint.mainwindow.MainWindow.openSample}}\pysiglinewithargsret{\sphinxbfcode{\sphinxupquote{openSample}}}{\emph{file\_path}}{}
Open a HDF5 sample file.
\begin{quote}\begin{description}
\item[{Parameters}] \leavevmode
\sphinxstyleliteralstrong{\sphinxupquote{file\_path}} (\sphinxhref{https://docs.python.org/3/library/stdtypes.html\#str}{\sphinxstyleliteralemphasis{\sphinxupquote{str}}}) \textendash{} Path to a file containing a sample.

\end{description}\end{quote}

\end{fulllineitems}

\index{openSequence() (mrsprint.mainwindow.MainWindow method)}

\begin{fulllineitems}
\phantomsection\label{\detokenize{autodoc/mrsprint/mrsprint:mrsprint.mainwindow.MainWindow.openSequence}}\pysiglinewithargsret{\sphinxbfcode{\sphinxupquote{openSequence}}}{\emph{file\_path}}{}
Open a python sequence file.

The file must contain a class SequenceExample, that contains
information about the sequence as RF pulses and the Gradient.
\begin{quote}\begin{description}
\item[{Parameters}] \leavevmode
\sphinxstyleliteralstrong{\sphinxupquote{file\_path}} (\sphinxhref{https://docs.python.org/3/library/stdtypes.html\#str}{\sphinxstyleliteralemphasis{\sphinxupquote{str}}}) \textendash{} Path to a file containing the sequence.

\end{description}\end{quote}

\end{fulllineitems}

\index{openSystem() (mrsprint.mainwindow.MainWindow method)}

\begin{fulllineitems}
\phantomsection\label{\detokenize{autodoc/mrsprint/mrsprint:mrsprint.mainwindow.MainWindow.openSystem}}\pysiglinewithargsret{\sphinxbfcode{\sphinxupquote{openSystem}}}{\emph{file\_path}}{}
Open a HDF5 system file.
\begin{quote}\begin{description}
\item[{Parameters}] \leavevmode
\sphinxstyleliteralstrong{\sphinxupquote{file\_path}} (\sphinxhref{https://docs.python.org/3/library/stdtypes.html\#str}{\sphinxstyleliteralemphasis{\sphinxupquote{str}}}) \textendash{} Path to a file containing the system.

\end{description}\end{quote}

\end{fulllineitems}

\index{plotSequence() (mrsprint.mainwindow.MainWindow method)}

\begin{fulllineitems}
\phantomsection\label{\detokenize{autodoc/mrsprint/mrsprint:mrsprint.mainwindow.MainWindow.plotSequence}}\pysiglinewithargsret{\sphinxbfcode{\sphinxupquote{plotSequence}}}{}{}
Use the data from a sequence to create the RF and Gradient plots.

\end{fulllineitems}

\index{quit() (mrsprint.mainwindow.MainWindow method)}

\begin{fulllineitems}
\phantomsection\label{\detokenize{autodoc/mrsprint/mrsprint:mrsprint.mainwindow.MainWindow.quit}}\pysiglinewithargsret{\sphinxbfcode{\sphinxupquote{quit}}}{}{}
Close the main window.

\end{fulllineitems}

\index{resizeTable2DEditor() (mrsprint.mainwindow.MainWindow method)}

\begin{fulllineitems}
\phantomsection\label{\detokenize{autodoc/mrsprint/mrsprint:mrsprint.mainwindow.MainWindow.resizeTable2DEditor}}\pysiglinewithargsret{\sphinxbfcode{\sphinxupquote{resizeTable2DEditor}}}{}{}
Change the size of the tableWidgets whenever a row/column is added/removed.

\end{fulllineitems}

\index{sampleSettingsChanges() (mrsprint.mainwindow.MainWindow method)}

\begin{fulllineitems}
\phantomsection\label{\detokenize{autodoc/mrsprint/mrsprint:mrsprint.mainwindow.MainWindow.sampleSettingsChanges}}\pysiglinewithargsret{\sphinxbfcode{\sphinxupquote{sampleSettingsChanges}}}{}{}
Reload sample parameters.

\end{fulllineitems}

\index{save() (mrsprint.mainwindow.MainWindow method)}

\begin{fulllineitems}
\phantomsection\label{\detokenize{autodoc/mrsprint/mrsprint:mrsprint.mainwindow.MainWindow.save}}\pysiglinewithargsret{\sphinxbfcode{\sphinxupquote{save}}}{\emph{file\_path}}{}
Create a HDF5 file to save current data.
\begin{quote}\begin{description}
\item[{Parameters}] \leavevmode
\sphinxstyleliteralstrong{\sphinxupquote{file\_path}} (\sphinxhref{https://docs.python.org/3/library/stdtypes.html\#str}{\sphinxstyleliteralemphasis{\sphinxupquote{str}}}) \textendash{} Path to the file where the data will be saved.

\end{description}\end{quote}

\end{fulllineitems}

\index{saveSample() (mrsprint.mainwindow.MainWindow method)}

\begin{fulllineitems}
\phantomsection\label{\detokenize{autodoc/mrsprint/mrsprint:mrsprint.mainwindow.MainWindow.saveSample}}\pysiglinewithargsret{\sphinxbfcode{\sphinxupquote{saveSample}}}{\emph{file\_path}}{}
Save a HDF5 sample file.
\begin{quote}\begin{description}
\item[{Parameters}] \leavevmode
\sphinxstyleliteralstrong{\sphinxupquote{file\_path}} (\sphinxhref{https://docs.python.org/3/library/stdtypes.html\#str}{\sphinxstyleliteralemphasis{\sphinxupquote{str}}}) \textendash{} Path to the file where the sample will be saved.

\end{description}\end{quote}

\end{fulllineitems}

\index{saveSystem() (mrsprint.mainwindow.MainWindow method)}

\begin{fulllineitems}
\phantomsection\label{\detokenize{autodoc/mrsprint/mrsprint:mrsprint.mainwindow.MainWindow.saveSystem}}\pysiglinewithargsret{\sphinxbfcode{\sphinxupquote{saveSystem}}}{\emph{file\_path}}{}
Save a HDF5 system file.
\begin{quote}\begin{description}
\item[{Parameters}] \leavevmode
\sphinxstyleliteralstrong{\sphinxupquote{file\_path}} (\sphinxhref{https://docs.python.org/3/library/stdtypes.html\#str}{\sphinxstyleliteralemphasis{\sphinxupquote{str}}}) \textendash{} Path to the file where the system will be saved.

\end{description}\end{quote}

\end{fulllineitems}

\index{setIndexLimits2DEditor() (mrsprint.mainwindow.MainWindow method)}

\begin{fulllineitems}
\phantomsection\label{\detokenize{autodoc/mrsprint/mrsprint:mrsprint.mainwindow.MainWindow.setIndexLimits2DEditor}}\pysiglinewithargsret{\sphinxbfcode{\sphinxupquote{setIndexLimits2DEditor}}}{\emph{nx}, \emph{ny}, \emph{nz}}{}
Set index limits for 2D editor.
\begin{quote}\begin{description}
\item[{Parameters}] \leavevmode\begin{itemize}
\item {} 
\sphinxstyleliteralstrong{\sphinxupquote{nx}} (\sphinxhref{https://docs.python.org/3/library/functions.html\#int}{\sphinxstyleliteralemphasis{\sphinxupquote{int}}}) \textendash{} Shape of data in the x dimension.

\item {} 
\sphinxstyleliteralstrong{\sphinxupquote{ny}} (\sphinxhref{https://docs.python.org/3/library/functions.html\#int}{\sphinxstyleliteralemphasis{\sphinxupquote{int}}}) \textendash{} Shape of data in the y dimension.

\item {} 
\sphinxstyleliteralstrong{\sphinxupquote{nz}} (\sphinxhref{https://docs.python.org/3/library/functions.html\#int}{\sphinxstyleliteralemphasis{\sphinxupquote{int}}}) \textendash{} Shape of data in the z dimension.

\end{itemize}

\end{description}\end{quote}

\end{fulllineitems}

\index{setSettingsPath() (mrsprint.mainwindow.MainWindow method)}

\begin{fulllineitems}
\phantomsection\label{\detokenize{autodoc/mrsprint/mrsprint:mrsprint.mainwindow.MainWindow.setSettingsPath}}\pysiglinewithargsret{\sphinxbfcode{\sphinxupquote{setSettingsPath}}}{}{}
\end{fulllineitems}

\index{showAxis3DView() (mrsprint.mainwindow.MainWindow method)}

\begin{fulllineitems}
\phantomsection\label{\detokenize{autodoc/mrsprint/mrsprint:mrsprint.mainwindow.MainWindow.showAxis3DView}}\pysiglinewithargsret{\sphinxbfcode{\sphinxupquote{showAxis3DView}}}{\emph{show=True}}{}
Show axis in 3D view.
\begin{quote}\begin{description}
\item[{Parameters}] \leavevmode
\sphinxstyleliteralstrong{\sphinxupquote{show}} (\sphinxhref{https://docs.python.org/3/library/functions.html\#bool}{\sphinxstyleliteralemphasis{\sphinxupquote{bool}}}) \textendash{} Informs if the axis should be visible. Default is True.

\end{description}\end{quote}

\end{fulllineitems}

\index{simulate() (mrsprint.mainwindow.MainWindow method)}

\begin{fulllineitems}
\phantomsection\label{\detokenize{autodoc/mrsprint/mrsprint:mrsprint.mainwindow.MainWindow.simulate}}\pysiglinewithargsret{\sphinxbfcode{\sphinxupquote{simulate}}}{}{}
Setup a simulation.

\end{fulllineitems}

\index{simulatorSettingsChanges() (mrsprint.mainwindow.MainWindow method)}

\begin{fulllineitems}
\phantomsection\label{\detokenize{autodoc/mrsprint/mrsprint:mrsprint.mainwindow.MainWindow.simulatorSettingsChanges}}\pysiglinewithargsret{\sphinxbfcode{\sphinxupquote{simulatorSettingsChanges}}}{}{}
Reload simulator parameters.

\end{fulllineitems}

\index{systemSettingsChanges() (mrsprint.mainwindow.MainWindow method)}

\begin{fulllineitems}
\phantomsection\label{\detokenize{autodoc/mrsprint/mrsprint:mrsprint.mainwindow.MainWindow.systemSettingsChanges}}\pysiglinewithargsret{\sphinxbfcode{\sphinxupquote{systemSettingsChanges}}}{}{}
Reload system parameters.

\end{fulllineitems}

\index{tabify2DEditor() (mrsprint.mainwindow.MainWindow method)}

\begin{fulllineitems}
\phantomsection\label{\detokenize{autodoc/mrsprint/mrsprint:mrsprint.mainwindow.MainWindow.tabify2DEditor}}\pysiglinewithargsret{\sphinxbfcode{\sphinxupquote{tabify2DEditor}}}{\emph{value}}{}
Tabify 2D editor depending on value.
\begin{quote}\begin{description}
\item[{Parameters}] \leavevmode
\sphinxstyleliteralstrong{\sphinxupquote{value}} (\sphinxhref{https://docs.python.org/3/library/functions.html\#bool}{\sphinxstyleliteralemphasis{\sphinxupquote{bool}}}) \textendash{} Informs if the tables in 2DEditor should be tabified or not.

\end{description}\end{quote}

\end{fulllineitems}

\index{translateGrid3DView() (mrsprint.mainwindow.MainWindow method)}

\begin{fulllineitems}
\phantomsection\label{\detokenize{autodoc/mrsprint/mrsprint:mrsprint.mainwindow.MainWindow.translateGrid3DView}}\pysiglinewithargsret{\sphinxbfcode{\sphinxupquote{translateGrid3DView}}}{}{}
Translate grids to correct position.

\end{fulllineitems}

\index{updateContext() (mrsprint.mainwindow.MainWindow method)}

\begin{fulllineitems}
\phantomsection\label{\detokenize{autodoc/mrsprint/mrsprint:mrsprint.mainwindow.MainWindow.updateContext}}\pysiglinewithargsret{\sphinxbfcode{\sphinxupquote{updateContext}}}{}{}
If context is changed, update interface.

\end{fulllineitems}

\index{updateDataFromTable2DEditor() (mrsprint.mainwindow.MainWindow method)}

\begin{fulllineitems}
\phantomsection\label{\detokenize{autodoc/mrsprint/mrsprint:mrsprint.mainwindow.MainWindow.updateDataFromTable2DEditor}}\pysiglinewithargsret{\sphinxbfcode{\sphinxupquote{updateDataFromTable2DEditor}}}{\emph{item\_changed}}{}
Update data from table in 2D editor.

\end{fulllineitems}

\index{updateExplorer() (mrsprint.mainwindow.MainWindow method)}

\begin{fulllineitems}
\phantomsection\label{\detokenize{autodoc/mrsprint/mrsprint:mrsprint.mainwindow.MainWindow.updateExplorer}}\pysiglinewithargsret{\sphinxbfcode{\sphinxupquote{updateExplorer}}}{}{}
Update explorer files for current context.

\end{fulllineitems}

\index{updateObserverData() (mrsprint.mainwindow.MainWindow method)}

\begin{fulllineitems}
\phantomsection\label{\detokenize{autodoc/mrsprint/mrsprint:mrsprint.mainwindow.MainWindow.updateObserverData}}\pysiglinewithargsret{\sphinxbfcode{\sphinxupquote{updateObserverData}}}{}{}
Reread data and apply updates.

\end{fulllineitems}

\index{updateObserverDataSize() (mrsprint.mainwindow.MainWindow method)}

\begin{fulllineitems}
\phantomsection\label{\detokenize{autodoc/mrsprint/mrsprint:mrsprint.mainwindow.MainWindow.updateObserverDataSize}}\pysiglinewithargsret{\sphinxbfcode{\sphinxupquote{updateObserverDataSize}}}{}{}
Reread data and apply updates, maybe need some rebuild.

\end{fulllineitems}

\index{updateObserverIndex() (mrsprint.mainwindow.MainWindow method)}

\begin{fulllineitems}
\phantomsection\label{\detokenize{autodoc/mrsprint/mrsprint:mrsprint.mainwindow.MainWindow.updateObserverIndex}}\pysiglinewithargsret{\sphinxbfcode{\sphinxupquote{updateObserverIndex}}}{}{}
Reread indexes and apply updates, does not need recreated anything.

\end{fulllineitems}

\index{updateSelectedCells() (mrsprint.mainwindow.MainWindow method)}

\begin{fulllineitems}
\phantomsection\label{\detokenize{autodoc/mrsprint/mrsprint:mrsprint.mainwindow.MainWindow.updateSelectedCells}}\pysiglinewithargsret{\sphinxbfcode{\sphinxupquote{updateSelectedCells}}}{\emph{value}, \emph{index}}{}
Update selected cells from parameter tree.
\begin{quote}\begin{description}
\item[{Parameters}] \leavevmode\begin{itemize}
\item {} 
\sphinxstyleliteralstrong{\sphinxupquote{value}} (\sphinxhref{https://docs.python.org/3/library/functions.html\#float}{\sphinxstyleliteralemphasis{\sphinxupquote{float}}}) \textendash{} Value to be updated in the table.

\item {} 
\sphinxstyleliteralstrong{\sphinxupquote{index}} (\sphinxhref{https://docs.python.org/3/library/functions.html\#int}{\sphinxstyleliteralemphasis{\sphinxupquote{int}}}) \textendash{} Index of selected type of data.

\end{itemize}

\end{description}\end{quote}

\end{fulllineitems}

\index{updateTableFromData2DEditor() (mrsprint.mainwindow.MainWindow method)}

\begin{fulllineitems}
\phantomsection\label{\detokenize{autodoc/mrsprint/mrsprint:mrsprint.mainwindow.MainWindow.updateTableFromData2DEditor}}\pysiglinewithargsret{\sphinxbfcode{\sphinxupquote{updateTableFromData2DEditor}}}{}{}
Update items of all 2D editor using 3d-array data and set color.

This method is called when the data shape is changed or when
the plane index is changed. Also, when the data index is changed.

\end{fulllineitems}

\index{updateViewFromData3DView() (mrsprint.mainwindow.MainWindow method)}

\begin{fulllineitems}
\phantomsection\label{\detokenize{autodoc/mrsprint/mrsprint:mrsprint.mainwindow.MainWindow.updateViewFromData3DView}}\pysiglinewithargsret{\sphinxbfcode{\sphinxupquote{updateViewFromData3DView}}}{}{}
Color 3D view based on 3D data.

\end{fulllineitems}


\end{fulllineitems}



\subsubsection{mrsprint.settings module}
\label{\detokenize{autodoc/mrsprint/mrsprint:module-mrsprint.settings}}\label{\detokenize{autodoc/mrsprint/mrsprint:mrsprint-settings-module}}\index{mrsprint.settings (module)}
Module responsible for the configuration of settings of all other modules.
\begin{description}
\item[{Authors:}] \leavevmode\begin{itemize}
\item {} 
Victor Hugo de Mello Pessoa \textless{}\sphinxhref{mailto:victor.pessoa@usp.br}{victor.pessoa@usp.br}\textgreater{}

\item {} 
Daniel Cosmo Pizetta \textless{}\sphinxhref{mailto:daniel.pizetta@usp.br}{daniel.pizetta@usp.br}\textgreater{}

\end{itemize}

\item[{Since:}] \leavevmode
2017/07/01

\end{description}

\begin{sphinxadmonition}{note}{\label{autodoc/mrsprint/mrsprint:index-3}Todo:}
Insert log inside functions.
\end{sphinxadmonition}
\index{Settings (class in mrsprint.settings)}

\begin{fulllineitems}
\phantomsection\label{\detokenize{autodoc/mrsprint/mrsprint:mrsprint.settings.Settings}}\pysiglinewithargsret{\sphinxbfcode{\sphinxupquote{class }}\sphinxcode{\sphinxupquote{mrsprint.settings.}}\sphinxbfcode{\sphinxupquote{Settings}}}{\emph{*args}, \emph{**kwargs}}{}
Bases: \sphinxcode{\sphinxupquote{PyQt4.QtGui.QMainWindow}}

Main window for settings.

\begin{sphinxadmonition}{note}{\label{autodoc/mrsprint/mrsprint:index-4}Todo:}
Include restore default settings.
Create a better check for values.
\end{sphinxadmonition}
\index{blockUnblockSignals() (mrsprint.settings.Settings method)}

\begin{fulllineitems}
\phantomsection\label{\detokenize{autodoc/mrsprint/mrsprint:mrsprint.settings.Settings.blockUnblockSignals}}\pysiglinewithargsret{\sphinxbfcode{\sphinxupquote{blockUnblockSignals}}}{\emph{value}}{}
Block or unblock signals defined by value.
\begin{quote}\begin{description}
\item[{Parameters}] \leavevmode
\sphinxstyleliteralstrong{\sphinxupquote{value}} (\sphinxhref{https://docs.python.org/3/library/functions.html\#bool}{\sphinxstyleliteralemphasis{\sphinxupquote{bool}}}) \textendash{} True to block, false otherwise.

\end{description}\end{quote}

\end{fulllineitems}

\index{check() (mrsprint.settings.Settings method)}

\begin{fulllineitems}
\phantomsection\label{\detokenize{autodoc/mrsprint/mrsprint:mrsprint.settings.Settings.check}}\pysiglinewithargsret{\sphinxbfcode{\sphinxupquote{check}}}{}{}
Evaluate if parameters are between its limits before saving them.
\begin{quote}\begin{description}
\item[{Returns}] \leavevmode
True if OK.

\item[{Return type}] \leavevmode
\sphinxhref{https://docs.python.org/3/library/functions.html\#bool}{bool}

\end{description}\end{quote}

\begin{sphinxadmonition}{note}{\label{autodoc/mrsprint/mrsprint:index-5}Todo:}
Reduce complexity and check all value limits.
\end{sphinxadmonition}

\end{fulllineitems}

\index{open() (mrsprint.settings.Settings method)}

\begin{fulllineitems}
\phantomsection\label{\detokenize{autodoc/mrsprint/mrsprint:mrsprint.settings.Settings.open}}\pysiglinewithargsret{\sphinxbfcode{\sphinxupquote{open}}}{\emph{file\_path}, \emph{set\_file=True}}{}
Open a HDF5 settings file.
\begin{quote}\begin{description}
\item[{Parameters}] \leavevmode\begin{itemize}
\item {} 
\sphinxstyleliteralstrong{\sphinxupquote{file\_path}} (\sphinxhref{https://docs.python.org/3/library/stdtypes.html\#str}{\sphinxstyleliteralemphasis{\sphinxupquote{str}}}) \textendash{} Path to a file containing the settings.

\item {} 
\sphinxstyleliteralstrong{\sphinxupquote{set\_file}} (\sphinxhref{https://docs.python.org/3/library/functions.html\#bool}{\sphinxstyleliteralemphasis{\sphinxupquote{bool}}}) \textendash{} set path file if true, otherwise just import.

\end{itemize}

\end{description}\end{quote}

\end{fulllineitems}

\index{openFile() (mrsprint.settings.Settings method)}

\begin{fulllineitems}
\phantomsection\label{\detokenize{autodoc/mrsprint/mrsprint:mrsprint.settings.Settings.openFile}}\pysiglinewithargsret{\sphinxbfcode{\sphinxupquote{openFile}}}{\emph{set\_file=True}}{}
Open a dialog to select the config file to be opened.

\end{fulllineitems}

\index{save() (mrsprint.settings.Settings method)}

\begin{fulllineitems}
\phantomsection\label{\detokenize{autodoc/mrsprint/mrsprint:mrsprint.settings.Settings.save}}\pysiglinewithargsret{\sphinxbfcode{\sphinxupquote{save}}}{\emph{file\_path}, \emph{set\_file=True}}{}
Save a HDF5 settings file.
\begin{quote}\begin{description}
\item[{Parameters}] \leavevmode\begin{itemize}
\item {} 
\sphinxstyleliteralstrong{\sphinxupquote{file\_path}} (\sphinxhref{https://docs.python.org/3/library/stdtypes.html\#str}{\sphinxstyleliteralemphasis{\sphinxupquote{str}}}) \textendash{} Path to a file containing the settings.

\item {} 
\sphinxstyleliteralstrong{\sphinxupquote{set\_file}} (\sphinxhref{https://docs.python.org/3/library/functions.html\#bool}{\sphinxstyleliteralemphasis{\sphinxupquote{bool}}}) \textendash{} set path file if true, otherwise just export.

\end{itemize}

\end{description}\end{quote}

\end{fulllineitems}

\index{saveFile() (mrsprint.settings.Settings method)}

\begin{fulllineitems}
\phantomsection\label{\detokenize{autodoc/mrsprint/mrsprint:mrsprint.settings.Settings.saveFile}}\pysiglinewithargsret{\sphinxbfcode{\sphinxupquote{saveFile}}}{\emph{set\_file=True}}{}
Open a dialog to select the config file to be saved.

\end{fulllineitems}


\end{fulllineitems}



\subsubsection{Module contents}
\label{\detokenize{autodoc/mrsprint/mrsprint:module-mrsprint}}\label{\detokenize{autodoc/mrsprint/mrsprint:module-contents}}\index{mrsprint (module)}
Magnetic resonance experiment simulator and visualization tool.
\begin{description}
\item[{Authors:}] \leavevmode\begin{itemize}
\item {} 
Daniel Cosmo Pizetta \textless{}\sphinxhref{mailto:daniel.pizetta@usp.br}{daniel.pizetta@usp.br}\textgreater{}

\item {} 
Victor Hugo de Mello Pessoa \textless{}\sphinxhref{mailto:victor.pessoa@usp.br}{victor.pessoa@usp.br}\textgreater{}

\end{itemize}

\item[{Since:}] \leavevmode
2015/07/01

\end{description}


\section{scripts}
\label{\detokenize{autodoc/scripts/modules:scripts}}\label{\detokenize{autodoc/scripts/modules::doc}}

\subsection{generate\_qrc module}
\label{\detokenize{autodoc/scripts/generate_qrc:generate-qrc-module}}\label{\detokenize{autodoc/scripts/generate_qrc::doc}}

\subsection{get\_version module}
\label{\detokenize{autodoc/scripts/get_version:get-version-module}}\label{\detokenize{autodoc/scripts/get_version::doc}}

\subsection{process\_icons module}
\label{\detokenize{autodoc/scripts/process_icons:process-icons-module}}\label{\detokenize{autodoc/scripts/process_icons::doc}}

\subsection{process\_ui module}
\label{\detokenize{autodoc/scripts/process_ui:process-ui-module}}\label{\detokenize{autodoc/scripts/process_ui::doc}}

\chapter{Downloads}
\label{\detokenize{download:downloads}}\label{\detokenize{download::doc}}
Here you will find documents and files to download.


\section{Download binaries - click-and-run}
\label{\detokenize{download:download-binaries-click-and-run}}
Binaries are for those do not wish to install any Python things.
We recommend them to the ones without any programming experience.
Download from links below.
\begin{itemize}
\item {} 
Portable Windows Binaries: coming soon!

\item {} 
Portable Linux Binaries: coming soon!

\item {} 
Portable Mac Binaries: coming soon!

\end{itemize}

Sou you can just download, decompress, click-and-run.


\section{Documentation}
\label{\detokenize{download:documentation}}
Links for latest available documentation.


\chapter{Changelog}
\label{\detokenize{changes:changelog}}\label{\detokenize{changes::doc}}

\section{v1.2}
\label{\detokenize{changes:v1-2}}\begin{itemize}
\item {} 
Add file explorer

\item {} 
Add about

\item {} 
Add docs to ReadTheDocs

\item {} 
Improved and new icons, and logo

\item {} 
Bug fixes

\end{itemize}


\section{v1.1}
\label{\detokenize{changes:v1-1}}\begin{itemize}
\item {} 
Fix docs, examples

\end{itemize}


\section{v1.0}
\label{\detokenize{changes:v1-0}}\begin{itemize}
\item {} 
First public API

\end{itemize}


\section{v0.6}
\label{\detokenize{changes:v0-6}}\begin{itemize}
\item {} 
New structure, revised files and screenshots

\item {} 
Fix doc style

\item {} 
Change nucleous to nucleus

\item {} 
Fix pyqtgraph imports

\end{itemize}


\section{v0.5}
\label{\detokenize{changes:v0-5}}\begin{itemize}
\item {} 
Add new extensions for HDF5 files for contexts

\item {} 
More examples of samples

\item {} 
Add simulation context

\item {} 
Bug fixes

\end{itemize}


\section{v0.4}
\label{\detokenize{changes:v0-4}}\begin{itemize}
\item {} 
Add sample examples

\item {} 
Bug fixes

\item {} 
Improvements on 2D editor and 3D plot

\end{itemize}


\section{v0.3}
\label{\detokenize{changes:v0-3}}\begin{itemize}
\item {} 
UI Improvements

\item {} 
Add git-lab CI

\item {} 
Add pylint

\item {} 
Open and save samples in HDF5

\item {} 
Bug fixes

\end{itemize}


\section{v0.2}
\label{\detokenize{changes:v0-2}}\begin{itemize}
\item {} 
Removing unnecessary things

\item {} 
Improvements in code

\item {} 
Docs with Sphinx

\end{itemize}


\section{v0.1}
\label{\detokenize{changes:v0-1}}\begin{itemize}
\item {} 
First version with binaries for Windows and Linux - Pyinstaller

\item {} 
Add tox

\item {} 
Add scripts for pyinstaller, ui, icons

\item {} 
2D editor and 3D view

\item {} 
Main window and toolbars

\end{itemize}


\chapter{Authors}
\label{\detokenize{authors:authors}}\label{\detokenize{authors::doc}}\begin{itemize}
\item {} 
\sphinxhref{mailto:daniel.pizetta@usp.br}{Daniel Cosmo Pizetta}

\item {} 
\sphinxhref{mailto:victor.pessoa@usp.br}{Victor Hugo de Mello Pessoa}

\end{itemize}


\chapter{License}
\label{\detokenize{license:license}}\label{\detokenize{license::doc}}

\section{Code - The MIT License}
\label{\detokenize{license:code-the-mit-license}}
Copyright (c) 2015-2018 Daniel Cosmo Pizetta

Permission is hereby granted, free of charge, to any person obtaining a copy
of this software and associated documentation files (the “Software”), to deal
in the Software without restriction, including without limitation the rights
to use, copy, modify, merge, publish, distribute, sublicense, and/or sell
copies of the Software, and to permit persons to whom the Software is
furnished to do so, subject to the following conditions:

The above copyright notice and this permission notice shall be included in
all copies or substantial portions of the Software.

THE SOFTWARE IS PROVIDED “AS IS”, WITHOUT WARRANTY OF ANY KIND, EXPRESS OR
IMPLIED, INCLUDING BUT NOT LIMITED TO THE WARRANTIES OF MERCHANTABILITY,
FITNESS FOR A PARTICULAR PURPOSE AND NONINFRINGEMENT. IN NO EVENT SHALL THE
AUTHORS OR COPYRIGHT HOLDERS BE LIABLE FOR ANY CLAIM, DAMAGES OR OTHER
LIABILITY, WHETHER IN AN ACTION OF CONTRACT, TORT OR OTHERWISE, ARISING FROM,
OUT OF OR IN CONNECTION WITH THE SOFTWARE OR THE USE OR OTHER DEALINGS IN
THE SOFTWARE.


\section{Images - Creative Commons Attribution International 4.0}
\label{\detokenize{license:images-creative-commons-attribution-international-4-0}}
Copyright (c) 2015-2018 Daniel Cosmo Pizetta

Creative Commons Corporation (“Creative Commons”) is not a law firm and does not provide legal services or legal advice. Distribution of Creative Commons public licenses does not create a lawyer-client or other relationship. Creative Commons makes its licenses and related information available on an “as-is” basis. Creative Commons gives no warranties regarding its licenses, any material licensed under their terms and conditions, or any related information. Creative Commons disclaims all liability for damages resulting from their use to the fullest extent possible.


\subsection{Using Creative Commons Public Licenses}
\label{\detokenize{license:using-creative-commons-public-licenses}}
Creative Commons public licenses provide a standard set of terms and conditions that creators and other rights holders may use to share original works of authorship and other material subject to copyright and certain other rights specified in the public license below. The following considerations are for informational purposes only, are not exhaustive, and do not form part of our licenses.
\begin{itemize}
\item {} 
\sphinxstylestrong{Considerations for licensors:} Our public licenses are intended for use by those authorized to give the public permission to use material in ways otherwise restricted by copyright and certain other rights. Our licenses are irrevocable. Licensors should read and understand the terms and conditions of the license they choose before applying it. Licensors should also secure all rights necessary before applying our licenses so that the public can reuse the material as expected. Licensors should clearly mark any material not subject to the license. This includes other CC-licensed material, or material used under an exception or limitation to copyright. \sphinxhref{http://wiki.creativecommons.org/Considerations\_for\_licensors\_and\_licensees\#Considerations\_for\_licensors}{More considerations for licensors}.

\item {} 
\sphinxstylestrong{Considerations for the public:} By using one of our public licenses, a licensor grants the public permission to use the licensed material under specified terms and conditions. If the licensor’s permission is not necessary for any reason\textendash{}for example, because of any applicable exception or limitation to copyright\textendash{}then that use is not regulated by the license. Our licenses grant only permissions under copyright and certain other rights that a licensor has authority to grant. Use of the licensed material may still be restricted for other reasons, including because others have copyright or other rights in the material. A licensor may make special requests, such as asking that all changes be marked or described. Although not required by our licenses, you are encouraged to respect those requests where reasonable. \sphinxhref{http://wiki.creativecommons.org/Considerations\_for\_licensors\_and\_licensees\#Considerations\_for\_licensees}{More considerations for the public}.

\end{itemize}


\chapter{Indices and tables}
\label{\detokenize{index:indices-and-tables}}\begin{itemize}
\item {} 
\DUrole{xref,std,std-ref}{genindex}

\item {} 
\DUrole{xref,std,std-ref}{modindex}

\item {} 
\DUrole{xref,std,std-ref}{search}

\end{itemize}


\renewcommand{\indexname}{Python Module Index}
\begin{sphinxtheindex}
\def\bigletter#1{{\Large\sffamily#1}\nopagebreak\vspace{1mm}}
\bigletter{m}
\item {\sphinxstyleindexentry{mrsprint}}\sphinxstyleindexpageref{autodoc/mrsprint/mrsprint:\detokenize{module-mrsprint}}
\item {\sphinxstyleindexentry{mrsprint.globals}}\sphinxstyleindexpageref{autodoc/mrsprint/mrsprint:\detokenize{module-mrsprint.globals}}
\item {\sphinxstyleindexentry{mrsprint.gui}}\sphinxstyleindexpageref{autodoc/mrsprint/mrsprint.gui:\detokenize{module-mrsprint.gui}}
\item {\sphinxstyleindexentry{mrsprint.gui.mrsprint\_rc}}\sphinxstyleindexpageref{autodoc/mrsprint/mrsprint.gui:\detokenize{module-mrsprint.gui.mrsprint_rc}}
\item {\sphinxstyleindexentry{mrsprint.gui.mw\_gradient}}\sphinxstyleindexpageref{autodoc/mrsprint/mrsprint.gui:\detokenize{module-mrsprint.gui.mw_gradient}}
\item {\sphinxstyleindexentry{mrsprint.gui.mw\_mrsprint}}\sphinxstyleindexpageref{autodoc/mrsprint/mrsprint.gui:\detokenize{module-mrsprint.gui.mw_mrsprint}}
\item {\sphinxstyleindexentry{mrsprint.gui.mw\_settings}}\sphinxstyleindexpageref{autodoc/mrsprint/mrsprint.gui:\detokenize{module-mrsprint.gui.mw_settings}}
\item {\sphinxstyleindexentry{mrsprint.mainwindow}}\sphinxstyleindexpageref{autodoc/mrsprint/mrsprint:\detokenize{module-mrsprint.mainwindow}}
\item {\sphinxstyleindexentry{mrsprint.sequence}}\sphinxstyleindexpageref{autodoc/mrsprint/mrsprint.sequence:\detokenize{module-mrsprint.sequence}}
\item {\sphinxstyleindexentry{mrsprint.sequence.sequence}}\sphinxstyleindexpageref{autodoc/mrsprint/mrsprint.sequence:\detokenize{module-mrsprint.sequence.sequence}}
\item {\sphinxstyleindexentry{mrsprint.settings}}\sphinxstyleindexpageref{autodoc/mrsprint/mrsprint:\detokenize{module-mrsprint.settings}}
\item {\sphinxstyleindexentry{mrsprint.simulator}}\sphinxstyleindexpageref{autodoc/mrsprint/mrsprint.simulator:\detokenize{module-mrsprint.simulator}}
\item {\sphinxstyleindexentry{mrsprint.simulator.plot}}\sphinxstyleindexpageref{autodoc/mrsprint/mrsprint.simulator:\detokenize{module-mrsprint.simulator.plot}}
\item {\sphinxstyleindexentry{mrsprint.simulator.simulator}}\sphinxstyleindexpageref{autodoc/mrsprint/mrsprint.simulator:\detokenize{module-mrsprint.simulator.simulator}}
\item {\sphinxstyleindexentry{mrsprint.subject}}\sphinxstyleindexpageref{autodoc/mrsprint/mrsprint.subject:\detokenize{module-mrsprint.subject}}
\item {\sphinxstyleindexentry{mrsprint.subject.sample}}\sphinxstyleindexpageref{autodoc/mrsprint/mrsprint.subject:\detokenize{module-mrsprint.subject.sample}}
\item {\sphinxstyleindexentry{mrsprint.system}}\sphinxstyleindexpageref{autodoc/mrsprint/mrsprint.system:\detokenize{module-mrsprint.system}}
\item {\sphinxstyleindexentry{mrsprint.system.gradient}}\sphinxstyleindexpageref{autodoc/mrsprint/mrsprint.system:\detokenize{module-mrsprint.system.gradient}}
\item {\sphinxstyleindexentry{mrsprint.system.magnet}}\sphinxstyleindexpageref{autodoc/mrsprint/mrsprint.system:\detokenize{module-mrsprint.system.magnet}}
\item {\sphinxstyleindexentry{mrsprint.system.rf}}\sphinxstyleindexpageref{autodoc/mrsprint/mrsprint.system:\detokenize{module-mrsprint.system.rf}}
\end{sphinxtheindex}

\renewcommand{\indexname}{Index}
\printindex
\end{document}